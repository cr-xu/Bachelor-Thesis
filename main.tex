\documentclass{include/thesisclass}
% Main File - Based on thesisclass.cls
% Comments are mostly in English
% ------------------------------------------------------------------------------
% Further files in folder:
%  - include/cmds.tex (for macros and additional commands)
%  - include/kitlogo.pdf (for titlepage)
%  - lit.bib (bibtex bibliography database)
%  - include/titlepage.tex (for layout of titelpage)
% ------------------------------------------------------------------------------
% Useful Supplied Packages:
% amsmath, amssymb, mathtools, bbm, upgreek, nicefrac,
% siunitx, varioref, booktabs, graphicx, tikz, multicol





%% -------------------------
%% |    Thesis Settings    |
%% -------------------------
% english or ngerman (new german für neue deutsche Rechtschreibung statt german)
\SelectLanguage{english}
% details on this thesis
\newcommand{\thesisauthor}{Chenran Xu}
\newcommand{\thesistopic}{Untersuchung der Langzeitstabilität des EDELWEISS Myon-Veto-Systems}
\newcommand{\thesisentopic}{Investigation of the long term stability of the EDELWEISS muon veto system}
\newcommand{\thesislongtopic}{Very long and very detailed description of the very interesting thesis topic (only necessary for pdfsubject tag).}
\newcommand{\thesisinstitute}{Institut für Experimentelle Kernphysik}
\newcommand{\thesisreviewerone}{Prof. Dr. Guido Drexlin}
\newcommand{\thesisreviewertwo}{Dr. Klaus Eitel}
\newcommand{\thesisadvisorone}{} % to use: enter names and uncomment in titlepg
\newcommand{\thesisadvisortwo}{}
\newcommand{\thesistimestart}{01.04.2015} % on titlepage
\newcommand{\thesistimeend}{30.09.2015} % on titlepage
\newcommand{\thesistimehandin}{30.10.2017} % on second page 'preamble'
\newcommand{\thesispagehead}{Bachelor Thesis: \thesisentopic} % page heading





%% ---------------------
%% |    PDF - Setup    |
%% ---------------------
% This information will appear embed into the PDF file as meta data, but will
% not be printed anywhere
\hypersetup
{
    pdfauthor={\thesisauthor},
    pdftitle={Bachelorarbeit: \thesistopic},
    pdfsubject={\thesislongtopic},
    pdfkeywords={kit,physik,bachelor,thesis,\thesisauthor}
}





%% --------------------------------------
%% |    Settings for Word Separation    |
%% --------------------------------------
% Help for separation:
% In German package the following hints are additionally available:
% "- = Additional separation
% "| = Suppress ligation and possible separation (e.g. Schaf"|fell)
% "~ = Hyphenation without separation (e.g. bergauf und "~ab)
% "= = Hyphenation with separation before and after
% "" = Separation without a hyphenation (e.g. und/""oder)

% Describe separation hints here:
\hyphenation
{
    über-nom-me-nen an-ge-ge-be-nen
    %Pro-to-koll-in-stan-zen
    %Ma-na-ge-ment  Netz-werk-ele-men-ten
    %Netz-werk Netz-werk-re-ser-vie-rung
    %Netz-werk-adap-ter Fein-ju-stier-ung
    %Da-ten-strom-spe-zi-fi-ka-tion Pa-ket-rumpf
    %Kon-troll-in-stanz
}





%% -----------------------
%% |    Main Document    |
%% -----------------------
\usepackage{lipsum} % for Lorem Ipsum text example
\usepackage{subcaption}
%\usepackage[notoc,notlof,notlot]{tocbibind}
\newcommand{\mvs}{muon-veto system}

\begin{document}
    % Titlepage and ToC
    \FrontMatter

    \input{include/titlepage}
    \input{include/preamble}

    \begingroup \let\clearpage\relax    % in order to avoid listoffigures and
    \tableofcontents                    % listoftables on new pages
    %\listoffigures
    %\listoftables
    \endgroup
    \cleardoublepage



    % Contents
    \MainMatter

    \chapter{Introduction}
    \label{chap:intro}
    Astrophysical and cosmological observations over the last decades indicate the existence of some non-baryonic dark matter. By analyzing the anisotropy of cosmic microwave bacground, the dark matter energy contribution is estimated to be 27\% of the universe \cite{Ade16}. Yet no knowledge of the particle consitituent of the dark matter is obtained.

A generic class of hypothetical particles, the Weakly Interacting Massive Particle (WIMP), is a prominent candidate for the dark matter. WIMPs are often assumed to have a mass of $\mathcal{O}$(\SI{100}{GeV}), with an interaction cross section with ordinary matter of the order of the weak interaction scale.

The EDELWEISS experiment aims to search for a direct signal of elastic scattering of WIMP on germanium nuclei. Due to the expected low rate of WIMP-nucleus scattering, the main challenge of the experiment is to understand and exclude possibly all the background events.

The detectors are surrounded by multiple layers of external shielding, which absorb and reject bacground radioactivity. Further backgrounds from the radioactivity of the shieding materials can be discriminated by the simultaneous readout of heat and ionization measurements. The remaining neutron bacground causes a nuclear recoil in detectors, which cannot be distinguished from a WIMP-signal. The neutrons are produced respectively from the cosmic-ray muons and natural radioactivity. To protect the detectors from the cosmic muon bacgrounds, EDELWEISS is located in the underground laboratory in Modane (Laboratoire Souterrain de Modane,LSM), where the muon flux is reduced to 5\,$\mathrm{\mu}$/$\mathrm{m}^{2}$/d \cite{Sch13a}. The remaining muons are tagged by a \mvs of 46 plastic scintillator modules.

Since the start of EDELWEISS experiment, the modules as well as the electronics have decayed significantly. The goal of this presented work is to estimate the stability of the muon veto system over long term measurements. Four extra scintillator modules installed in 2010 are equipped with LEDs to moniter the stability of the system.

In chapter \ref{chap:theo} the case of dark matter with focus on WIMPs is discussed, followed by a brief description of the general setup of EDELWEISS experiment. In chapter \ref{chap:muon} the setup and the working principle of \mvs is described. First, the LED events are analysed to estimate the long term stability of these four modules in chapter \ref{chap:ana_led}. Second, the muon events are selected for the analysis of all modules in chapter \ref{chap:ana_muon}. Additionaly, the effective threshold of each module is determined to estimate the change of detection efficiency of the \mvs.

    %\input{./chap/chapter1.tex}
    \chapter{Search for WIMPs with the EDELWEISS experiment}
    \label{chap:theo}
    % chapter2.tex
%motivation of the analysis of long term measurement~~ Dark Matter->WIMP->EDEILWEISS->Background

Nowadays, the search for dark matter becomes one of the central topics in astroparticle physics.  Numerous experiments aim to search for dark matter. In this chapter, observational evidences of dark matter are given, followed by a description of particle candidates of dark matter with the focus on WIMPs. EDELWEISS is one of the experiments to directly search for dark matter. The general setup and working principle is described.

%enregieanteil durch CMB ,lambda-cdm,...
%outline for this chapter
%%%%%%%%%%%%%%%%%%%%%%%%%%%%%%%%%%%%%%%%%%%%%%%%%%%%%%%%%%%%%%%%%%%%%%%%%%%%%%%%
%%%%%%%%%%%%%%%%%%%%%%%%%%%%%%%%%%%%%%%%%%%%%%%%%%%%%%%%%%%%%%%%%%%%%%%%%%%%%%%%
\section{Evidences of dark matter}
In 1933, while studying the velocity dispersion of galaxies inside the Coma galaxy cluster, F. Zwicky inferred the existence of some kind of unseen mass, which he called \textit{dunkle Materie} (dark matter). Since then, his idea was supported by numerous observations on different scales -- e.g.\ Cosmic Microwave Background(CMB, \cite{Pla16}), the Bullet Cluster.
The Bullet Cluster (1E 0657-56) consists of two clusters, which collided around $\SI{100}{Myrs}$ ago. Using gravitational lensing and X-Ray analysis, it is found that the two galaxy concentrations have moved ahead of their plasma clouds, which indicates the existence of weak-interacting dark matter \cite{Clo06}.
In the following a more detailed description of the evidence of dark matter in galaxies is given.
\subsection*{Galaxy rotation curve}
Taking a simplified model where almost all the galactic mass is inside a radius R, the rotation velocity of an object at large distance from the galactic centre can be approximated:
\begin{equation}
  v(r)=\sqrt{\frac{GM}{r}}
\end{equation}
M denotes the galactic mass, and r the distance to the galactic centre with $\mathrm{r}>\mathrm{R}$. Therefore, the velocity is expected to behave as $v \approx r^{-1/2}$ at large distances according to Kepler's law. However, observation of flat rotation curves shows discrepancy from the expectation. In 1980, an extensive study of 21 galaxies suggested that most stars in spiral galaxies have roughly the same orbital velocity, which implies the existence of some kind of unseen matter \cite{Rub80}.
\begin{figure}[ht]
  \centering
  \includegraphics[width=0.75\textwidth{}]{./fig/rotation_curve.png}
  \caption{ An example of a galaxy rotation curve (NGC 6503). The observed rotation curve is plotted together with the individual components. Extracted from \cite{Rub80}.}.
  \label{fig:rotation-curve}
\end{figure}
Fig \ref{fig:rotation-curve} shows an example of the observed rotation curve of one spiral galaxy. The contribution of baryonic matter (disk and gas) to the velocity decreases with distance, whereas the DM-halo one rises. Together they lead to the observed flat curve at large radius.

It should also be mentioned that there are alternative theories to explain the problem of galaxy rotation curves, such as Modified Newtonian Dynamics (MOND, \cite{Mil83}). Although MOND successfully solves the problem in galactic scale, it cannot explain the observations at larger scales.
%%%%%%%%%%%%%%%%%%%%%%%%%%%%%%%%%%%%%%%%%%%%%%%%%%%%%%%%%%%%%%%%%%%%%%%%%%%%%%%%
%%%%%%%%%%%%%%%%%%%%%%%%%%%%%%%%%%%%%%%%%%%%%%%%%%%%%%%%%%%%%%%%%%%%%%%%%%%%%%%%
%%%%%%%%%%%%%%%%%%%%%%%%%%%%%%%%%%%%%%%%%%%%%%%%%%%%%%%%%%%%%%%%%%%%%%%%%%%%%%%%

\section{The WIMP as a dark matter particle candidate}
The $\mathrm{\Lambda}$ cold dark matter ($\mathrm{\Lambda}$CDM) model is a parametrization of the Big Bang model and successfully explains the evolution of the universe. It is therefore often referred to as the standard model of cosmology. As the name suggests, the $\mathrm{\Lambda}$CDM model contains as most prominent contribution a cosmological constant (denoted with $\mathrm{\Lambda}$) and cold dark matter, which means that dark matter mostly consists of non-relativistic particles. Also, dark matter is electrically uncharged and mostly collisionless. The DM particles only interact with themselves and other particles through gravitation and weak force. They have to be stable on a cosmological time scale, otherwise they would not exist with such abundance nowadays.

In the Standard Model (SM), no particle satisfies all the properties above. The SM is thus to be extended. There are many hypothetical particles as potential dark matter candidates, e.g.\ axions, sterile neutrinos, WIMPs. The axion is a hypothetical elementary particle to solve the strong CP problem \cite{Pec77}. Sterile neutrinos are right-handed neutrinos that only interact via gravitation. They would be a candidate of warm dark matter if their mass is in the keV range \cite{Dre13}. The dark matter candidate of interest in this work is called WIMP (weakly interacting massive particle), a generic class of hypothetical particles.

The WIMP is expected to have mass of a few GeV to a few TeV and to interact weakly and gravitationally.  For sufficiently high temperatures, like in the early universe, the WIMPs are constantly produced and annihilated. As the temperature drops, the WIMPs eventually cease to interact and the particle density remains roughly constant. A promising candidate of WIMPs is the so-called lightest supersymmetric particle (LSP) of the supersymmetric model (SUSY). SUSY is an extension of the standard model where each SM-particle has a SUSY-partner which differs only by a half-integer spin. In many models the LSP turns out to be a neutralino, which is the mixture of four SUSY-particles.

Due to the low interaction cross section, WIMPs are extremely hard to detect. They can be detected through different methods. They can be produced by the collision of SM-particles. WIMPs can also be detected indirectly by measuring the SM-particles produced in self-annihilation of dark matter. Lastly, they can be directly detected by observation of WIMP-nucleus scattering like in the EDELWEISS experiment. For detailed descriptions of the DM direct-detection experiments, see \cite{Und15}.

%%%%%%%%%%%%%%%%%%%%%%%%%%%%%%%%%%%%%%%%%%%%%%%%%%%%%%%%%%%%%%%%%%%%%%%%%%%%%%%%
%%%%%%%%%%%%%%%%%%%%%%%%%%%%%%%%%%%%%%%%%%%%%%%%%%%%%%%%%%%%%%%%%%%%%%%%%%%%%%%%
%%%%%%%%%%%%%%%%%%%%%%%%%%%%%%%%%%%%%%%%%%%%%%%%%%%%%%%%%%%%%%%%%%%%%%%%%%%%%%%%

\section{The EDELWEISS Experiment}
  \label{edw}
  The EDELWEISS experiment is dedicated to detect the scattering of WIMPs on ordinary matter at cryogenic temperature. In order to achieve the expected sensitivity down to \SI{e-9}{pb}, the main challenge is to exclude all the backgrounds induced by radioactivity or cosmic rays. The general setup of the experiment and the possible backgrounds are summarized in section \ref{sec:edw-exp}. The remaining backgrounds can be discriminated by measurements of two channels of the signal. This working principle of germanium bolometers is briefly described in section \ref{edw-ge}.
  The problematic muon-induced neutrons, which can hardly be distinguished from the WIMP-signal, are described in detail in chapter \ref{chap:muon}.
\subsection{Experimental setup and backgrounds of EDELWEISS-III}
  \label{sec:edw-exp}
  The EDELWEISS experiment is located in the underground laboratory of Modane (\textit{Laboratoire Souterrain de Modane}, LSM). Under 1780 meters of rock, the cosmic muon flux is reduced by more than a factor \num{e6} to a remaining rate of 5\,$\mathrm{\mu}$/$\mathrm{m}^{2}$/d \cite{Sch13a}.
  The remaining throughgoing muons are tagged with an active \mvs, which is the outermost layer of the setup (see fig.\ref{fig:exp-setup}). A detailed description and working principle are given in chapter \ref{chap:muon}.

  \begin{figure}[ht]
    \centering
    \includegraphics[width=0.75\textwidth{}]{./fig/exp_setup.png}
    \caption{\textbf{Schematic view of the EDELWEISS experimental setup}.
    In the center are the Ge-bolometers hosted in a cryostat. The cryostat is surrounded by a lead shield, a PE shield and an active muon veto system to minimize the backgrounds. Extracted from \cite{Kef16}.}
    \label{fig:exp-setup}
  \end{figure}


  The next layer is a polyethylene (PE) shield of about \SI{50}{cm} thickness to attenuate the neutron flux from the radioactivity of rock and experiment materials. The fast neutron flux with neutron energies above \SI{1}{MeV}, which produces similar recoils as from WIMPs, is reduced by 5 to 6 orders of magnitude. \\%\ref{backgrounds...}
  Next to the PE shield is a lead shield of 20 cm thickness to reduce the ambient $\upgamma$ background. The natural lead contains radioactive isotopes, e.g.\,\ce{^{210}Pb}, \ce{^{238}U} and \ce{^{232}Th}, which also contribute to the background. To reduce its natural radioactivity, the innermost \SI{2}{cm} of the shield is made of Roman lead discovered in a sunken ship. The \ce{^{210}Pb} has a half-life of $T_{1/2}=\SI{22}{years}$, so that its abundance is decreased by two orders of magnitude \cite{Sch13a}.
  Another source of external background is the \ce{^{222}Rn} gas in the air which is a decay product of \ce{^{238}U}.

  The upper part with the cryostat is installed in a clean room. The space between the lead shield and the cryostat is flushed with filtered air to minimize the radon level.
  The upper part of the shieldings are mounted on rails and can be opened in halves to access the cryostat and electronics. Additional layers of PE and lead shields are installed inside the cryostat to reduce the background induced by electronics and cables.

  The cryostat is a \ce{^{3}He}/\ce{^{4}He} dilution refrigerator made of low-radioactivity materials. The detectors are enclosed in five thermal screens and the temperature decreases from room temperature over \SI{100}{K}, \SI{40}{K}, \SI{4}{K}, \SI{1}{K} to \SI{10}{mK}. In standard operations, the temperature of the detectors is tuned to $T=(18.000 \pm 0.002)\,\si{mK}$. More details of the setup can be found in \cite{Arm17}.

\subsection{Working principle of Ge Bolometer}
  \label{edw-ge}
  The bolometers used in the EDELWEISS experiment are made of high-purity monocrystalline germanium. They are equipped with aluminium ring electrodes and glued with 2 Neutron Transmutation Doped (NTD) sensors.

  The thermalized phonon signals are measured via the change of resistance of the NTD Ge sensors. The small temperature rise resulted by an energy deposit $E_{\mathrm{rec}}$ is
  \begin{equation}
      \Delta T = \frac{E_{\mathrm{rec}}}{C(T)}
  \end{equation}
  by which $C(T)$ is the total heat capacity of the germanium crystal and two NTD sensors. The temperature dependency of the resistance is given by
  \begin{equation}
    R(T)=R_{0}exp\sqrt{\frac{T_{0}}{T}}
  \end{equation}
  with characteristic constants $R_{0}=\mathcal{O}(\SI{0.1}{\ohm})$ and $T_{0}=\mathcal{O}(\SI{1}{K})$. At the operating temperature of \SI{18}{mK}, the resistance becomes a few \si{\mega\ohm}. The NTD sensors are biased with a square modulated current and the resistance change is obtained by change of the voltage.%\cite{??}

  For each event, the ionization energy $E_{\mathrm{ion}}$ is measured simultaneously. Electron-hole pairs are produced in the germanium crystal for an energy deposit above \SI{2.96}{eV}. The produced charge carriers are drifted to the biased electrodes and collected.

  The discrimination between electron recoils and nuclear recoils is based on the ionization yield $Q$, defined as the fraction of ionization energy and recoil energy:
  \begin{equation}
    Q=\frac{E_{\mathrm{ion}}}{E_{\mathrm{rec}}}
  \end{equation}
  Since the WIMPs and neutrons scatter off nuclei, the required energy to produce a pair of charge carriers is higher than the one of electron recoils. Most energy deposited by nuclear recoils is directly transmitted to phonons, which leads to a generally smaller ionization yield than electron recoils.

  The heat and ionization channels are calibrated with the \SI{356}{keV} line of \ce{^{133}Ba}, which induces electron recoils. With the ionization yield of electron recoils set to 1, the neutron ionization yield is determined with a neutron calibration \cite{Dis01}:
  \begin{equation}
    Q_{\mathrm{n}}=0.16\cdot(E_{\mathrm{rec}}[\si{keV}])^{0.18}
  \end{equation}
  With combination of the heat and ionization measurements, the electron recoils can be distinguished from the nuclear recoils. Therefore, the remaining problematic background is neutrons, mostly produced in muon-induced showers or muon-nuclear interactions.

    \chapter{Muon detection in EDELWEISS experiment}
    \label{chap:muon}
    % chapter3.tex
% muon-veto system  Muon interaction, energy deposit, setup of muon veto, working principle, available data,
Despite the rock overburden of LSM which reduces the cosmic muon flux by 6 orders of magnitude, the remaining muons can produce neutrons and mimic WIMP signals. These muons are tagged by an active veto system. The general setup and the working principle of the system is described in this chapter. The description is mainly based on the doctoral thesis of C. K{\'{e}}f{\'{e}}lian \cite{Kef16}.


%%%%%%%%%%%%%%%%%%%%%%%%%%%%%%%%%%%%%%%%%%%%%%%%%%%%%%%%%%%%%%%%%%%%%%%%%%%%%%%%
%%%%%%%%%%%%%%%%%%%%%%%%%%%%%%%%%%%%%%%%%%%%%%%%%%%%%%%%%%%%%%%%%%%%%%%%%%%%%%%%

\section{Setup of the \mvs}
\label{sec:muon-setup}

The \mvs{} is the outermost layer of shieldings and covers a surface of $\SI{100}{m^{2}}$. As shown in the fig. \ref{fig:muon-setup}, it is made of 46 plastic scintillator modules. The modules are labelled from 1 to 48. Each wall is labelled according to the orientation. The western wall is named "Nemo", which is the name of the neighbour experiment in LSM.
The muon-veto is divided in two levels, the upper level made of 30 modules is located in a clean room which hosts the cryostat and the detectors. The lower level has 16 modules. As described in section \ref{sec:edw-exp}, the upper level is mounted on rails and can be opened in two parts to grant access to electronics and the cryostat.

\begin{figure}[ht!]
  \centering
  \includegraphics[width=0.5\textwidth{}]{./fig/Veto.png}
  \caption{Schematic view of the \mvs. Each wall of the system is labelled according to its geometric orientation in the laboratory. }.
  \label{fig:muon-setup}
\end{figure}

To cover the gap resulted from an extension of the cryogenic support line, additional modules M7, M8, M15 and M16 were installed in 2010. The four extra modules are equipped with LEDs to monitor the stability of the system. M7 and M8 have 3 LEDs along their axis, each of M15 and M16 has one LED installed in the middle. M7 and M8 are $\SI{2.1}{m}$ long. M15 and M16 are around $\SI{1}{m}$ long and cover only partly the opening of the upper part.

All modules have a width of $\SI{65}{cm}$ and a thickness of $\SI{5}{cm}$. Their lengths vary from $\SI{2}{m}$ to $\SI{4}{m}$.
Due to the opening for electronics and the shorter length of some modules, the overall geometric coverage is 98\%. However, a muon going through the gap can partly be detected via the particle showers induced by it or at its exit hitting another module.

A group of four Photomultiplier Tubes (PMT) is installed at each module end. Each PMT group is individually biased with a high voltage (HV). The HV values are typically set around $\SI{-1500}{V}$ and only seldom changed over years when an abnormal behaviour was observed in a module. To ensure that the system is fully closed while operating, two lasers measure the position of two halves of the upper part every 15 minutes. One measures the distance from the western wall to M6, the other from the eastern wall to M8. The gap width is calculated by subtracting the two distances.


\section{Working principle of the \mvs}
\label{sec:muon-working}


\subsection{Muon energy deposit in the scintillator modules}
The average muon energy at LSM is $<E_{\upmu}>_{\mathrm{LSM}}\approx \SI{260}{GeV}$ \cite{Klu13}. The high energy muons deposit ~$\SI{2}{MeV\per\cm}$ in the muon-veto modules according to the Bethe formula. Since the scintillator modules have a thickness of 5cm, the muon energy deposit in a module is typically above $\SI{10}{MeV}$. Therefore, the muon events can be separated from the background events with energy deposit normally lower than $\SI{4}{MeV}$, which reduces the dead-time of the experiment.
The stochastic process of muon energy deposit can be described by a Landau distribution \cite{Lan44}. Such distribution is asymmetric and has a long tail towards the high energy region. To avoid the contribution of large energy deposit from the long tail, the most probable value (MPV) is usually taken to characterize the distribution.
The total energy deposit of muon is also dependent on its path length in a module. The spectrum is thus smeared due to the different orientation of modules and the angular distribution of the muon flux. Most muons at LSM have small zenith angle, therefore the muons deposit minimal energy in top and bottom modules and the track length is of the order of the module thickness.
It is also possible that a muon goes through the edge of a module, which is called a \textit{grazing muon}. In such case, the muon traverses only partly the module thickness and deposits lower energy.

\subsection{The electronic readout chain}
The data acquisition of the \mvs{} is independent from the bolometer. The muon-veto is only rarely switched off when operation in the clean room is taking place.
When a muon goes through the scintillator modules, it deposits energy via different processes and produces scintillation light. The photons reflect in the module and are guided to the PMT groups. In a PMT group, the photons are then converted to electrons and amplified to a measurable electric signal. Once the signal amplitude is over the trigger threshold, the signal is integrated in the Analog-to-Digital-Converter (ADC) card to obtain the total energy deposit of a muon in a module. At the mean time, the Time-to-Digital-Converter (TDC) card stores the time of the signal. If there is a coincidence of both PMT groups in a module within a $\SI{100}{ns}$ time window, all non-zero signals of the \mvs{} are stored as one event. After the triggering, there is a dead-time of $\tau=\SI{0.16}{ms}$ when no events can be detected \cite{Sch13b}. The trigger threshold is set to $\SI{150}{mV}$ to ensure the detection efficiency of low energy events without introducing too much dead-time.

\subsection{Position-dependent light output}
In addition to the fluctuation of the muon energy deposit, the light output is also dependent on the position of the interaction in the scintillator modules. Since the light is guided up to $\SI{4}{m}$ to the PMT group, which is much larger than the attenuation length, the light measured by PMT decreases exponentially with the path length $d$. The relation can be approximately determined by the Beer-Lambert law:
\begin{equation}
  I(d)=I_{0}\cdot e^{-\frac{d}{\mathrm{\Lambda}_{\mathrm{eff}}}}
\end{equation}
The $\Lambda_{\mathrm{eff}}$ denotes the effective attenuation length and is a detector specific constant.
The scintillator modules in EDELWEISS were previously used in the KARMEN experiment. The effective attenuation length was measured to be $\mathrm{\Lambda} \approx \SI{600}{cm}$ in 1997/1998 \cite{Rei98}. However, the modules have aged since then. Some of the effects are radiation damages and decrease of the transparency.

\begin{figure}[ht!]
  \centering
  \includegraphics[width=0.6\textwidth{}]{./fig/pos-dependent.png}
  \caption{Light measured in the north and south PMT groups of Module 5 and the sum of them. The data are fitted with exponential curves. Extracted from \cite{Hab04}.}
  \label{fig:pos-dependent}
\end{figure}

In 2003/2004, the attenuation lengths of two $\SI{4}{m}$ modules were measured. Fig.\ \ref{fig:pos-dependent} shows the measured signal in M5 for two individual PMT groups and the sum of them. As shown in the figure, the light yield varies by a factor of 2 from the near end to the far end.
For M5 $\Lambda_{\mathrm{eff}}$ was around $\SI{340}{cm}$ and for M1 around $\SI{200}{cm}$ \cite{Hab04}. The measurement shows that the effective attenuation length of scintillator modules has decreased significantly since production, which leads to a decrease of discrimination efficiency for low energy events. This also motivates the importance to analyze the long term stability of the \mvs{}.


\subsection{Available information of the muon-veto events}

The measured events are stored in data files and combined to so-called Runs. Each Run contains up to 99 files, where each file stores 8 hours measurement. This means, one Run usually contains data of one month measurement time. The data in each Run file are converted to a \textit{KData} file. \textit{KData} is a ROOT-based \cite{Bru97} data structure and analysis toolkit developed at KIT. It combines the muon-veto data and the bolometer data for coincidence studies \cite{Cox12}.
The data branches relevant in the context of this work and available for the analysis are listed below:
\begin{itemize}
  \item \textbf{ADC: }
  When triggered, the integrated signal in each PMT group is stored in ADC units. The HVs of each PMT group are calibrated before the experiment to ensure a uniform gain of each module. Since the modules have aged individually, the correspondence between ADC channels and energy in MeV varies from module to module. There is also a digital conversion threshold. For an event with energy deposit below this value, the ADC is not stored and set to -1. The digital threshold is typically 120\,ADC channels.

  \item \textbf{TDC: }
  The arrival time of a signal in each PMT group is stored. By subtracting the two TDC values in one module, the event position can be reconstructed. In the presented work, the TDC values are only used to probe if a PMT group is triggered.


  \item \textbf{PC Time: }
  The time of each event is stored in units of seconds. For the muon-veto events to be compared with bolometer events, an additional timestamp in 10$\,\si{\upmu s}$ precision is saved. However, such precision is of no interest in the analysis of the long term stability. Therefore, the event time in seconds is used in the following analysis.

  \item \textbf{DistanceEst, DistanceNemo:}
  As described before, the gap size of the upper part of the system is measured every 15 minutes. DistanceEst stores the distance from the western wall to M8 (eastern half of the muon-veto) and DistanceNemo stores the distance from western wall to M6 (western-most module of the upper muon-veto). For each event, the distances obtained from the last laser measurement are stored.

  \item \textbf{IsLEDFired: }
  The LEDs installed on the extra top modules fire every eight hours to monitor the stability of the system. When the event is caused by LED firing, the boolean value IsLEDFired is set to true, allowing a distinction between LED events and other events.
\end{itemize}



%available values

    \chapter{Long term behavior of data acquired with LEDs}
    \label{chap:ana_led}
    % chapter4.tex
% LED Chapter (Muon?)
As described in Chapter \ref{chap:muon}, the four extra modules added in 2010 covering the gap of veto system are equipped with LEDs. M7 and M8 have three LEDs: one at the center and the other two at two ends. M15 and M16 both have one LED installed at the center. The LEDs send out pulses every eight hours. The LED data are used to perform a stability controll of the $\upmu$-veto system. They are clearly defined comparing to muon induced events, therefore the LED events are good probe to estimate the long term stability of these four modules.
\section{Data selection}
The data of muon-veto Run70 to Run138 are used to analyze the aging effect of the veto system. This corresponds to a date from 24.08.2010 to 28.03.2017. When converting the raw data to ROOT-format, the events induced by LED firing are flagged. Therefore, they are easily separated from other events.
The LEDs fire three times every day. Each LED fires 60 pulses in one minute and they fire one after another, which also allows a separation of signals from different LEDs in M7 and M8.



\section{Long term stability}
The LEDs are fixed on the modules, so the energy spectrum is not smeared by the position dependent light readout. Also, the LEDs are supposed to have constant light output over short time. Thus the spectrum can be fitted with a gaussian function to get the average ADC values of several series.
To increase the statistical power of a single point, events of nine shot series (three days) are combined to perfom a gaussian fit. An example of such fit is illustrated in fig.\ref{fig:gaussian-fit}.

\begin{figure}[ht!]
  \centering
  \includegraphics[width=0.6\textwidth{}]{./fig/gaussianM8.pdf}
  \caption{Example of an gaussian fit to nine LED fire series in Module 8, north PMT group. The spectrum is fitted with log likelihood method in ROOT.}
  \label{fig:gaussian-fit}
\end{figure}

The mean ADC values obtained from each gaussian fit are plotted over time. A change of these values over time could be due to various effects, e.g. decrease of the LED light output, aging of scintillator modules, problem of the PMTs or readout elctronics. To identify the contribution of different factors, the values are plotted separately for two PMT groups and three LEDs (for M7 and M8). Linear regressions are made for each data set, see fig.\ref{fig:M8LED}. The lines with different color represent the data from different LEDs.

As can be seen in the figure, the mean ADC values of two off-center LEDs differ about 1000 canals from the far end to the near end.





\begin{figure}[ht]
  \centering
  \includegraphics[width=0.9\textwidth{}]{./fig/M8LED.pdf}
  \caption{\textbf{The ADC values of LED signals over time in Module 8.}} The energy deposit of LED signals in ADC channels from Run70 to Run138 are plotted separately for 2 PMT groups (north in upper chart, south in lower chart). The trend of ADC values of different LEDs over time are approximated by linear fits: the green line (LED north), the blue line (LED middle), and the red line (LED south). For clarity reasons, the signals of the south LED are decreased by 500 channels in upper chart and increased by 500 in lower chart.
  \label{fig:M8LED}
\end{figure}

    \chapter{Determination of the long term stability using muon events }
    \label{chap:ana_muon}
    
% Muon events, threshold analysis
Analysis of LED events can only be applied on the four extra top modules. To determine the stability of the total \mvs, ageing of all modules must be investigated. In this chapter, the muon-events are selected and analysed. Additionally, the effective threshold of each module is determined using an independent method, which allows to estimate the change of the detection efficiency of the modules.

\section{Analysis of muon events}

\subsection{Selection of muon-veto data}
All the data from Run70 (Aug 2010) to Run138 (Mar 2017) are investigated in the following analysis. Since the goal is to analyse the stability of muon detection in modules, several cuts are applied on the data to ensure a proper acquisition condition of the system over the investigated time period and to decrease background events.\\
First, data taken when the \mvs{} was not fully closed are rejected. As described in sec.\,\ref{sec:muon-setup}, two lasers measure the position of the upper part of the muon-veto every 15 minutes. Time periods when the gap size deviates more than 5 cm from the closed configuration are cut, since the \mvs{} is either fully closed or fully opened. When the system is opened, the detection efficiency of through-going muons decreases and the background rate increases largely, which will lead to an increase of low energy events.

When the \mvs{} is closed, most muons go through at least 2 modules. Therefore a coincidence in 2 distinct modules is required for selecting muon events. An energy deposit with full information in each module is required, i.e.\,both TDCs and ADCs must have non-zero values. It respectively reduces events caused by secondary particles or radioactive backgrounds as they mostly deposit less energy than muons. By applying this cut criterium, the event rate in one module is reduced to the order of $\SI{10}{events/day}$.

\subsection{HV stability}
The HV applied on the modules over the investigated time period from year 2010 and 2017 are first analysed. During the time period, the HV of 25 scintillator modules are not changed. In March 2013, the HV on 21 modules were slightly increased to compensated the decrease of event rate due to the ageing effect. For example in Module 32, the HVs on the both PMT groups were increased from -1700\,V to -1710\,V (see fig.\,\ref{fig:HV_32}).

\begin{figure}[htb!]
  \centering
  \includegraphics[width=0.9\textwidth{}]{./fig/HVPlot_M32.pdf}
  \caption{HV applied on M32 from year 2010 to 2017. The HV applied on the module is negative and the plot shows the absolute value of the HV. The blue points are the set value of the HV and the black points are the actual measured HV. In March 2013, the set HV was increased from -1700\,V to -1710\,V for both PMT groups.}
  \label{fig:HV_32}
\end{figure}

For all modules, the result shows that the measured HV values only have small fluctuation from the demand values. Therefore, the HV is stable for all modules over the total time period.

\subsection{Data analysis}
As mentioned in section \ref{sec:muon-working}, the spectrum of muon energy deposit in modules can be described by a Landau distribution. The ADC values of muon events in a given time period are fitted with a Landau distribution and the obtained MPVs in different time periods are used to analyse the long term behaviour of the \mvs.  Fig.\,\ref{fig:Landau_M6} shows an example of the Landau fit. Since the muon event rate is low, the obtained ADC values in a two months period are combined to perform each fit.

\begin{figure}[ht]
  \centering
  \includegraphics[width=0.6\textwidth{}]{./fig/LandauFitM6.pdf}
  \caption{Example of a Landau fit in M6. The fit uses a log likelihood method in ROOT. The muon-induced events are collected over a period of two months.}
  \label{fig:Landau_M6}
\end{figure}

The MPVs obtained from the Landau fit are plotted over time (see fig.\ \ref{fig:MPV}). Some time periods with too few entries due to system shutoffs are excluded, since no reliable Landau fit can be performed. It is to be noticed that the data points have rather large uncertainty, which is partially due to the smeared spectrum. As explained before, the light output is strongly position dependent and decreases exponentially with the path length from the interaction point to the PMT group. Furthermore, there are some sudden changes of the MPVs. The reason could be the restart of the \mvs{} after operations and unstable data acquisition.

\begin{figure}[hp!]
  \centering
  \begin{subfigure}{0.6\linewidth}
    \centering
    \includegraphics[width=\linewidth{}]{./fig/M6mpv.pdf}
    \caption{}
    \label{fig:MPV_M6}
  \end{subfigure}
  \begin{subfigure}{0.6\linewidth}
    \centering
    \includegraphics[width=\linewidth{}]{./fig/M44mpv.pdf}
    \caption{}
    \label{fig:MPV_M44}
  \end{subfigure}
  \caption{MPVs over time with linear fits in two example modules. The two upper figures show the MPVs in M6 (top module) and the two lower figures show the MPVs in M44 (bottom module). See \ref{tab:mpv} for values of the gradients.}
  \label{fig:MPV}
\end{figure}

%
%\begin{figure}[htb]
%  \centering
%  \includegraphics[width=0.6\textwidth{}]{./fig/M8mpv.pdf}
%  \caption{MPVs over time with linear fit in M8.}
%  \label{fig:Landau_M8}
%\end{figure}

\begin{table}[hb!]
  \centering
  \caption{Slopes of the linear regressions of MPVs in example modules M6, M8 and M44. The statistical uncertainty is obtained from the fit program in ROOT. }
  \label{tab:mpv}
  \begin{tabular}{c c c c}
  \toprule
  Module & \multicolumn{2}{c}{slope in channels/month} \\
         & ADC N & ADC S \\
  \midrule
  M6  & $-5.4\pm1.3$ & $-4.9\pm1.1$ \\
  M8  & $-5.3\pm0.5$ & $-6.0\pm0.5$ \\
  M44 & $-5.8\pm3.7$ & $-3.7\pm1.1$ \\
  \bottomrule
  \end{tabular}
\end{table}

The change of the MPVs are again approximated by a linear regression and the obtained slopes are listed in tab.\ \ref{tab:mpv}. It can be seen that the slopes for two ends in a single module don't vary much from each other, which means that the ageing of plastic scintillator is, as expected, almost symmetric and the electronics of two PMT groups may have also aged similarly.

The value obtained in M8 is also listed for comparison with the result from LED data $\Delta{}_{\mathrm{LED\,S}}=\SI{-4.2}{channels\per month}$ and $\Delta{}_{\mathrm{LED \,N}}=\SI{-1.8}{channels\per month}$. The decrease of the MPVs is larger than the one of ADC mean values of middle and northern LED events and lies in the uncertainty region of southern LED events. The abnormal increase of ADC mean values of LED S in M8 (see tab.\ref{tab:led}) is not found in the data obtained from muon events. Thus it can be concluded that the problem is merely due to the behaviour of the LED but not the scintillator module. \\
Despite the various effects that may lead to a sudden change of the ADC value, the MPVs of the Landau spectrum in all 46 modules show a general decrease of several hundred ADC channels since 2010.



\section{Determination of the detection efficiency}
The analysis above shows that the signal height recorded with the PMT groups has decreased remarkably since the start of the experiment. However, it doesn't provide a direct measure of the change of the muon detection efficiency. To estimate the detection efficiency, the knowledge of the trigger threshold of each module is needed. Before the experiment, the scintillator modules have been calibrated using cosmic muons \cite{Hab04}. Due to the ageing effects of the modules, the trigger threshold must be determined again. As the muon flux in the underground laboratory is reduced significantly, the cosmic muons cannot be used for calibration anymore. In this section, an alternative approach to determine the trigger threshold is described. The detection efficiencies in different time periods are derived using the obtained threshold values.

\subsection{Effective trigger threshold}
As discussed in section \ref{sec:muon-working}, there are two thresholds in the measurement of the \mvs{}. One is a digital conversion threshold, which is typically set to 120\,ADC channels. The influence of the digital threshold on the muon detection efficiency can be neglected, because the MPVs are typically above 1000\,ADC channels and the Landau distribution falls quickly towards the low energy region.

The trigger threshold, on the other hand, is relevant for the following analysis. It is a hardware threshold which is set to 150\,mV. Only pulses with an amplitude above this value trigger the data acquisition. Although the hardware threshold is set to the same voltage in each module, the value in ADC channels depends on the gain of the individual PMT groups. Thus the effective trigger threshold in term of energy deposit is individual for each module. \\
Additionally, it depends on the shape of measured pulse, as the ADC value is the integral of a signal. For example, a flat pulse has a larger energy deposit than a steep pulse with same amplitude. The pulse shape also depends on the light propagation and is thus position dependent. Thus, a calibration source would be needed to determine the position dependency of the trigger threshold, which is not performed in this work.

Averaging over the module length and taking the resolution into account, the effective trigger threshold is expected to be a curve instead of a step function. The efficiency of a module can be given as a function of energy, which is low for small energy deposit and is expected to be 100\% in the high energy region. It is similar to the behaviour of an error function:
\begin{equation}
  \mathrm{erf}(x)=\frac{2}{\sqrt{\pi}} \int_{0}^{x} \! e^{-t^{2}}\, \mathrm{d}t
\end{equation}
It describes the convolution of a Heaviside function $\Theta(x)$ with a Gaussian distribution. By shifting and rescaling, the error function is able to imitate the behaviour of the actual detection efficiency curve at the threshold. It is characterised by two parameters, the effective threshold $E_{\mathrm{thr}}$ and the standard deviation $\sigma$. The effective trigger threshold is defined as the ADC value where detection efficiency is 50\% and the $\sigma$ describes the smearing effect of the spectrum.

\subsubsection*{A Method to determine the effective trigger threshold}

When a module is triggered, all the non-zero signals in the \mvs{} are stored. It is therefore possible to analyse the the events for a certain module $i$ even if they don't trigger the module $i$. For the analysis of one PMT group X in module $i$, its ADC values of all the events, that have been triggered by a full coincidence in another module $j$, are stored in histogram H1. In contrast to the condition of muon selection, which requires both non-zero TDC values, the trigger condition of the PMT group X used in the following analysis requires only that the TDC on the same side has a non-zero value. Requiring both TDC channels will introduce a dependency on the PMT group on the other side, while the trigger threshold is determined separately for two ends of a module. The events with a non-zero TDC value, i.e.\, have passed the threshold, are stored in histogram H2. The trigger efficiency of the PMT group X is determined by the division of H2 and H1.

\begin{figure}[ht]
  \centering
  \begin{subfigure}{0.7\linewidth}
    \includegraphics[width=\linewidth{}]{./fig/M6AdcNorth2Histo.pdf}
    \caption{Example of the spectrum with and without trigger condition of ADC N in M6. The data points (blue) are events which satisfy the trigger condition in the same end (Northern PMT group). The spectrum of all events (black) are scaled by a factor of 4.}
    \label{fig:2HistoM6}
  \end{subfigure}
  \begin{subfigure}{0.7\linewidth}
    \includegraphics[width=\linewidth{}]{./fig/M6AdcNortheff_late.pdf}
    \caption{Histogram of events triggering the PMT group on the same side divided by all events with error bars (black) and the error function fit (red).}
    \label{fig:eff_lateM6}
  \end{subfigure}
  \caption{Method to determine the effective trigger threshold of a PMT group exemplified for ADC North of Module 6.}
  \label{fig:threshold_example}
\end{figure}

Fig.\,\ref{fig:threshold_example} shows an example of this method using the events in Module 6. The events that triggered one PMT group are plotted with all events. As expected, the detection efficiency increases with ADC values and goes toward 100\%. As can be seen in the figure, the increase is slightly asymmetric: it is more rapid at first and is then flatter. The estimation with an error function is therefore conservative for rather small energy deposits and optimistic for higher energy events.

The effective trigger threshold in ADC units are calculated for each PMT group in different time periods. To increase statistics, several runs are combined to perform the error function fit. Run 70-79 (year 2010) are investigated for the early time period and Run 124-138 (year 2015-2017) for the late time period. The results are listed in tab.\,\ref{tab:efficiency_short} in the following section.


\subsection{Detection efficiency of a module}
In addition to the above criteria, events with a coincidence in adjacent modules are excluded. These events have a higher probability to be induced by secondary particles or grazing muons. They mostly have small energy deposits, where the efficiency is also small due to the trigger threshold, and can lead to a decrease of the derived total detection efficiency.

\begin{figure}[h!]
  \begin{subfigure}{0.5\linewidth}
    \includegraphics[width=\linewidth{}]{./fig/70M6CorrectedLandau.png}
  \end{subfigure}
  \begin{subfigure}{0.5\linewidth}
    \includegraphics[width=\linewidth{}]{./fig/124M6CorrectedLandau.png}
  \end{subfigure}
  \caption{Spectrum of muon-induced events in M6 in earlier runs (left) and latest runs (right). The spectra are fitted with a Landau distribution with an error function (red). The corrected spectrum (black, dotted) is determined by dividing the error function.}
  \label{fig:detection_M6}
\end{figure}

The muon energy deposit can be described with a Landau distribution, whereas the measured energy deposit is dependent on the response of the individual modules. Therefore, the ADC spectrum is expected to be a Landau distribution multiplied by an error function. The parameters of the error function are calculated above in the determination of the effective trigger threshold. The selected data are fitted with this function and the corrected spectrum is given by a Landau distribution with the obtained parameters from the fit (see fig. \ref{fig:detection_M6}). The efficiency can be determined by the integral of the spectrum with and without the error function.


\begin{table}[htb!]
  \caption{The trigger threshold and detection efficiency of PMT groups in selected modules on different sides of the \mvs{} (for the full table see appendix \ref{sec:appendix}). The MPV in this table denotes the most probable value of the Landau distribution. Parameters of the error function are given as Threshold and $\sigma{}_{\mathrm{erf}}$. The total efficiency is estimated via the product of the efficiency value $\epsilon_{50\%\mathrm{MPV}}$ of each PMT group, which is determined by the integration from half of the MPV energy (See text for more details). For M15, see also fig.\,\ref{fig:fail_M15}.}
  \label{tab:efficiency_short}
  \begin{tabular}{c c c c c c c c c}
    \toprule
    Module & End & MPV & Threshold & $\sigma{}_{\mathrm{erf}}$ & $\epsilon_{20\%}$ & $\epsilon_{50\%\mathrm{MPV}}$ & $\epsilon_{\mathrm{MPV}}$ & $\epsilon_{\mathrm{tot}, 50\%\mathrm{MPV}}$ \\
           &     & \multicolumn{3}{|c|}{in ADC channels} &   \\
    \midrule
    M6     & N & 1401 & 532 & 737 & 0.86 & 0.90 & 0.97 & \multirow{2}{*}{0.77}\\
    early  & S & 1256 & 615 & 712 & 0.83 & 0.86 & 0.95 &\\
    M6     & N & 872 & 707 & 627 & 0.67 & 0.71 & 0.85 & \multirow{2}{*}{0.53}\\
    late   & S & 975 & 746 & 416 & 0.60 & 0.75 & 0.94\\
    \midrule
    M15    & Nemo & 2674 & 560 & 143 & 0.99 & 1.00 & 1.00 & \multirow{2}{*}{1.00}\\
    early  & Est & 1292 & 186 & 13 & 1.00 & 1.00 & 1.00\\
    M15    & Nemo & 1713 & 1879 & 758 &  &  &  & \multirow{2}{*}{0.59}\\
    late   & Est & 1284 & 958 & 280 & 0.73 & 0.77 & 0.99 & \\
    \midrule
    M17    & B & 1200 & 515 & 112 & 0.99 & 1.00 & 1.00 & \multirow{2}{*}{0.94}\\
    early  & H & 1043 & 100 & 740 & 0.88 & 0.94 & 0.98 &\\
    M17    & B & 1457 & 855 & 384 & 0.82 & 0.90 & 0.99 & \multirow{2}{*}{0.80} \\
    late   & H & 1359 & 671 & 611 & 0.72 & 0.89 & 0.98 & \\
    \midrule
    M36    & Est & 1452 & 912 & 803 & 0.61 & 0.80 & 0.93 & \multirow{2}{*}{0.67} \\
    early  & Nemo & 1346 & 821 & 568 & 0.79 & 0.83 & 0.96 &\\
    M36    & Est & 1201 & 1114 & 624 & 0.48 & 0.62 & 0.85 & \multirow{2}{*}{0.45}\\
    late   & Nemo & 1373 & 1060 & 531 & 0.61 & 0.72 & 0.93 & \\
    \bottomrule
  \end{tabular}
\end{table}

However, integrating the total histogram starting from 0 will not give a reliable result. As the efficiency falls to 0 in low energy regions, this low-statistics part of the corrected spectrum will have large contribution and significantly reduce the total efficiency. Additionally, the fraction of muons with energy deposit well below the MPV is expected to be low in reality. Therefore, it is reasonable to integrate the spectrum from a certain energy value.\\
Three efficiency values with different integration start points are given for each PMT group. $\epsilon_{20\%}$ integrates from the energy where the entries are equal to 20\% of the entries at MPV. $\epsilon_{\mathrm{MPV}}$ integrates from the MPV and gives an optimistic estimation. Therefore, the total efficiency of a module is given by the product of $\epsilon_{50\%\mathrm{MPV}}$ in two PMT groups, for which the spectrum is integrated from half of the energy at MPV. The results for selected modules are listed in tab.\,\ref{tab:efficiency_short}, the complete table is given in appendix \ref{sec:appendix}.

For some of the PMT groups, the spectrum cannot be fitted properly (for an example see fig.\,\ref{fig:fail_M15}). The efficiency is estimated by the other PMT group in the same module. If the fit method cannot be applied for both PMT groups, the total detection efficiency is given by the mean values of other modules on the same side.


\begin{figure}[ht!]
  \centering
  \includegraphics[width=0.7\textwidth{}]{./fig/M15fail.pdf}
  \caption{Example of a failed fit of PMT Nemo in M15 in Run124-138. The efficiency is given by the other PMT group in the same module.}
  \label{fig:fail_M15}
\end{figure}

As expected, the MPV decreased and the effective trigger threshold increased in the time period of seven years, which lead to a loss of detection efficiency. Most modules show a relative decrease of about 30\%. Although the relative change is significant, the absolute values of detection efficiency should only be used as lower limits, since they are strongly underestimated as discussed below. An analysis in 2010 using a similar method obtained the result $\epsilon_{\mathrm{M6}}=0.88$ and $\epsilon_{\mathrm{M36}}=0.96$ \cite{Nie10}, showing that the efficiency values derived here are indeed too conservative.

The underestimation results from the following facts. First of all, there are still contributions of secondary particles or backgrounds to the low energy region despite the cuts applied. The obtained MPVs derived from the fit are thus smaller and it leads to a smaller detection efficiency.

Secondly, the gain of two PMT groups are considered as uncorrelated for the matter of simplification, while they are correlated in reality. As shown in section \ref{sec:muon-working}, the light yield measured in the near PMT group is larger than the one in the far PMT group. This leads to an overestimation for near end and underestimation for far end. Since the efficiency decreases drastically in the low energy region, the total detection efficiency is expected to be underestimated. However, the effect of correlation cannot be implemented due to the low statistics of muon events.

%why is eff so low

Additionally, the muons passing through the \mvs{} have a second chance to be detected when leaving the system. Also, they can partly be detected by measuring the particle showers. Last but not least, the grazing muons, which go through adjacent modules and deposit energy well below MPV, are possible to be detected in both modules. To conclude, the absolute value of detection efficiency derived here should be treated as lower limit and the total efficiency of the \mvs{} is higher than the one of individual modules.

%The result shows that the detection efficiency for most of the modules has decreased significantly. Despite that the rate of $\upmu{}$-induced WIMP-like signals is estimated to be low, it can limit the sensitivity of WIMP search if the efficiency of \mvs{} drops. To retain a high detection efficiency as the start of the experiment, the HV applied on individual PMT groups should be increased. By doing so, the gain of PMT groups increases, corresponding to a higher MPV in ADC units and lower effective threshold.

\subsection{Conclusion and comparison to earlier works}
In this chapter, the effective trigger threshold of each PMT group is determined. The detection efficiency of individual modules is derived using the selected muon events and the obtained threshold parameters. With the obtained detection efficiency of a single module, the probability for a muon going through 2 modules to be detected by the \mvs{} can be given as
\begin{equation}
  P_{\mathrm{total}}=\frac{1}{2}\cdot \sum\limits_{i} \left( P_{\mathrm{geo,i}}\cdot P_{i} +(1-P_{\mathrm{geo,i}}\cdot P_{i})\cdot \sum\limits_{j\ne i} P_{\mathrm{geo,ij}\cdot P_{j}}  \right)-P_{\mathrm{gap}},
\end{equation}
where $P_{i}$ is the detection efficiency of a single module $i$. Since the direction of muons are not equally distributed, the normalised probability that a muon goes through the module $i$ is given by $P_{\mathrm{geo,i}}$. $P_{\mathrm{geo,ij}}$ denotes the probability that the muon goes through both modules $i$ and $j$. Also, there is a small probability that a muon goes through the gap of the \mvs{}, which is given by $P_{\mathrm{gap}}$. \\
However, the detection efficiency of the total \mvs{} is not derived as the geometric coefficients are not available for this work. These parameters can be determined by a simulation of the muon flux and then the total detection efficiency can be caculated using the above equation. \\
Here, an example of the combined detection efficiency of two modules is described. Consider a muon going through M6 (top) and M46 (bottom), which probably hits the bolometers and can potentially induce a neutron signal. The detection efficiencies of M6 and M46 in early and late stage are
\begin{align}
  \epsilon_{6,\mathrm{early}} &= 0.77 & \epsilon_{6,\mathrm{late}} &= 0.53 \\
  \epsilon_{46,\mathrm{early}} &= 0.58 & \epsilon_{6,\mathrm{late}} &= 0.38.
\end{align}
Thus the probabilities that the muon is detected in both modules are
\begin{align}
  \epsilon_{6+46,\mathrm{early}} &= 0.89 \\
  \epsilon_{6+46,\mathrm{late}} &= 0.74.
\end{align}

The relative change of detection efficiency using combination M6 and M46 is thus 17\%. It it also expected that the change of the detection efficiency of the \mvs{} is smaller than the change of a single module. \\
The total detection efficiency of the \mvs{} was determined in 2010 to be $\epsilon_{\mathrm{total}} \geq 0.949$ using a similar method \cite{Nie10}. Therefore, as discussed before, the result obtained here gives a reliable lower limit on the actual value.

It should be mentioned that the detection efficiency of the \mvs{} can also estimated using different methods. For example, coincidences between the \mvs{} and the bolometers can be studied. As the muon-induced bolometer events can be distinguished from other backgrounds by their multiplicity and energy deposit, these events can be selected. The detection efficiency can thus be derived by searching for the coincidences in the \mvs{} in a certain time window. The detection efficiency of the \mvs{} is determined to be $\epsilon_{\mathrm{total}} \geq 0.93$ (90\% C.L.) in 2015 \cite{Kef16}.

It can be concluded that the detection efficiency for most of the modules has decreased significantly and the obtained value gives a lower limit on the actual value. The relative change of the detection efficiency is still significant.
Despite that the rate of $\upmu{}$-induced WIMP-like signals is estimated to be low, it can limit the sensitivity of WIMP search if the efficiency of the \mvs{} drops. \\
To retain a detection efficiency as high as at the start of the experiment, the HV applied on individual PMT groups should be increased. By doing so, the gain of PMT groups increases, corresponding to a higher MPV in ADC units and lower effective threshold. However, the decrease of the detection efficiency cannot be infinitely compensated by a higher HV, as the applied HV has an upper limit.\\
As mentioned before, the 46 scintillator modules were previously used in the KARMEN experiment and have an age over 20 years \cite{Rei98}. The analysis above shows that almost all of them have aged significantly over the last 7 years. Although they have shown a sufficient performance during the EDELWEISS experiment, they are not suitable to be used as a muon shielding for upgrades or further experiments requiring a higher sensitivity.






%blahblah

    \chapter{Conclusions}
    \label{conclusion}



    %\emptychapter[3]{ROOT Routines}     % usage: \emptychapter[page displayed
                                        %        in toc]{name of the chapter}



    % appendix for more or less interesting calculations
    \Appendix
    \chapter*{\appendixname} \addcontentsline{toc}{chapter}{\appendixname}
    % to make the appendix appear in ToC without number. \appendixname =
    % Appendix or Anhang (depending on chosen language)
    \section{Detection Efficiency}
some text!
%tables of efficiency
\small
\begin{longtable}{c c c c c c c c c}
  \caption{Slopes of the linear regressions of MPVs in example modules M6 and M44. The statistical uncertainty is obtained from the fit program in ROOT. } \\
  %\label{tab:mpv-full}
  \toprule
  Module & End & MPV & Threshold & $\sigma{}_{\mathrm{erf}}$ & $\epsilon_{20\%}$ & $\epsilon_{50\%\mathrm{MPV}}$ & $\epsilon_{\mathrm{MPV}}$ & $\epsilon_{\mathrm{tot}, 50\%\mathrm{MPV}}$ \\
         &     & \multicolumn{3}{|c|}{in ADC channels} &   \\
  \midrule
  \endfirsthead
  M1 & ADC[0] & 1036 & 939 & 690 & 0.66 & 0.65 & 0.80 & \multirow{2}{*}{0.4}\\
     & ADC[1] & 827 & 191 & 2193 & 0.69 & 0.71 & 0.75 & \\
  M2 & ADC[0] & 625 & 786 & 1622 & 0.56 & 0.60 & 0.66\\
     & ADC[1] & 214 & 829 & 1905 & 0.53 & 0.55 & 0.57\\
  M3 & ADC[0] & 647 & 494 & 98 & 0.88 & 0.83 & 1.00\\
     & ADC[1] & 918 & 635 & 136 & 0.90 & 0.90 & 1.00\\
  M4 & ADC[0] & 352 & 880 & 194 & 0.01 & 0.01 & 0.01\\
     & ADC[1] & 355 & 890 & 157 & 0.00 & 0.00 & 0.00\\
  M5 & ADC[0] & 1197 & 554 & 865 & 0.79 & 0.84 & 0.93\\
     & ADC[1] & 1108 & 645 & 607 & 0.78 & 0.83 & 0.94\\
  \midrule
  M6 & ADC[0] & 1401 & 532 & 737 & 0.86 & 0.90 & 0.97\\
     & ADC[1] & 1256 & 615 & 712 & 0.83 & 0.86 & 0.95\\
  M7 & ADC[0] & 1080 & 190 & 366 & 0.97 & 0.99 & 1.00\\
     & ADC[1] & 941 & 202 & 223 & 0.98 & 0.99 & 1.00\\
  M8 & ADC[0] & 1443 & 638 & 155 & 0.99 & 0.99 & 1.00\\
     & ADC[1] & 2217 & 251 & 101 & 0.98 & 1.00 & 1.00\\
  M9 & ADC[0] & 1320 & 751 & 791 & 0.71 & 0.82 & 0.93\\
     & ADC[1] & 3000 & 100 & 1282 & 0.86 & 0.97 & 1.00\\
  M10 & ADC[0] & 650 & 476 & 910 & 0.68 & 0.76 & 0.84\\
     & ADC[1] & 1410 & 100 & 989 & 0.87 & 0.93 & 0.98\\
  \midrule
  M11 & ADC[0] & 1133 & 685 & 409 & 0.82 & 0.86 & 0.98\\
     & ADC[1] & 952 & 493 & 176 & 0.94 & 0.97 & 1.00\\
  M12 & ADC[0] & 881 & 660 & 120 & 0.92 & 0.86 & 1.00\\
     & ADC[1] & 1140 & 395 & 146 & 0.97 & 1.00 & 1.00\\
  M13 & ADC[0] & 799 & 122 & 1184 & 0.79 & 0.85 & 0.90\\
     & ADC[1] & 1331 & 298 & 784 & 0.84 & 0.94 & 0.98\\
  M14 & ADC[0] & 1118 & 836 & 489 & 0.73 & 0.75 & 0.92\\
     & ADC[1] & 1138 & 100 & 1729 & 0.77 & 0.82 & 0.87\\
  M15 & ADC[0] & 2674 & 560 & 143 & 0.99 & 1.00 & 1.00\\
     & ADC[1] & 1292 & 186 & 13 & 1.00 & 1.00 & 1.00\\
  \midrule
  M16 & ADC[0] & 2481 & 267 & 91 & 0.99 & 1.00 & 1.00\\
     & ADC[1] & 2451 & 296 & 103 & 0.98 & 1.00 & 1.00\\
  M17 & ADC[0] & 1200 & 515 & 112 & 0.99 & 1.00 & 1.00\\
     & ADC[1] & 1043 & 100 & 740 & 0.88 & 0.94 & 0.98\\
  M18 & ADC[0] & 893 & 692 & 1254 & 0.65 & 0.69 & 0.78\\
     & ADC[1] & 2184 & 517 & 153 & 0.97 & 1.00 & 1.00\\
  M19 & ADC[0] & 1207 & 587 & 146 & 0.97 & 0.98 & 1.00\\
     & ADC[1] & 1799 & 533 & 146 & 0.99 & 1.00 & 1.00\\
  M20 & ADC[0] & 2133 & 1352 & 1343 & 0.55 & 0.72 & 0.87\\
     & ADC[1] & 1476 & 687 & 639 & 0.74 & 0.90 & 0.98\\
  M21 & ADC[0] & 1117 & 701 & 841 & 0.65 & 0.79 & 0.91\\
     & ADC[1] & 1219 & 529 & 1024 & 0.73 & 0.84 & 0.92\\
  M22 & ADC[0] & 442 & 1156 & 349 & 0.09 & 0.08 & 0.14\\
     & ADC[1] & 396 & 1008 & 227 & 0.01 & 0.01 & 0.01\\
  M25 & ADC[0] & 349 & 100 & 745 & 0.78 & 0.83 & 0.87\\
     & ADC[1] & 1261 & 708 & 1196 & 0.69 & 0.77 & 0.87\\
  M26 & ADC[0] & 488 & 355 & 866 & 0.70 & 0.77 & 0.83\\
     & ADC[1] & 1202 & 1005 & 1117 & 0.56 & 0.70 & 0.82\\
  M27 & ADC[0] & 1139 & 1110 & 810 & 0.60 & 0.61 & 0.77\\
     & ADC[1] & 1035 & 1104 & 941 & 0.58 & 0.58 & 0.71\\
  M28 & ADC[0] & 339 & 879 & 1238 & 0.51 & 0.55 & 0.59\\
     & ADC[1] & 958 & 1075 & 856 & 0.45 & 0.60 & 0.76\\
  M29 & ADC[0] & 696 & 1289 & 836 & 0.31 & 0.41 & 0.56\\
     & ADC[1] & 1334 & 1253 & 1137 & 0.59 & 0.62 & 0.76\\
  M30 & ADC[0] & 1166 & 1269 & 998 & 0.56 & 0.57 & 0.70\\
     & ADC[1] & 816 & 1054 & 1184 & 0.52 & 0.57 & 0.67\\
  M31 & ADC[0] & 1142 & 1164 & 847 & 0.58 & 0.59 & 0.75\\
     & ADC[1] & 504 & 969 & 1574 & 0.54 & 0.58 & 0.63\\
  M32 & ADC[0] & 466 & 1387 & 1456 & 0.41 & 0.46 & 0.51\\
     & ADC[1] & 340 & 1338 & 1263 & 0.37 & 0.41 & 0.46\\
  M33 & ADC[0] & 1031 & 593 & 264 & 0.86 & 0.93 & 1.00\\
     & ADC[1] & 1404 & 755 & 321 & 0.91 & 0.94 & 1.00\\
  M34 & ADC[0] & 1091 & 1459 & 1129 & 0.39 & 0.52 & 0.67\\
     & ADC[1] & 1157 & 1404 & 852 & 0.49 & 0.49 & 0.66\\
  M35 & ADC[0] & 1125 & 1723 & 1192 & 0.41 & 0.42 & 0.54\\
     & ADC[1] & 1016 & 1557 & 1039 & 0.42 & 0.41 & 0.52\\
  M36 & ADC[0] & 1452 & 912 & 803 & 0.61 & 0.80 & 0.93\\
     & ADC[1] & 1346 & 821 & 568 & 0.79 & 0.83 & 0.96\\
  M37 & ADC[0] & 1411 & 730 & 505 & 0.77 & 0.90 & 0.99\\
     & ADC[1] & 1329 & 746 & 425 & 0.85 & 0.89 & 0.99\\
  M38 & ADC[0] & 1446 & 1247 & 1167 & 0.61 & 0.65 & 0.79\\
     & ADC[1] & 1223 & 1293 & 700 & 0.54 & 0.54 & 0.75\\
  M39 & ADC[0] & 1570 & 967 & 734 & 0.74 & 0.81 & 0.95\\
     & ADC[1] & 1156 & 876 & 484 & 0.75 & 0.75 & 0.92\\
  M40 & ADC[0] & 1208 & 1098 & 913 & 0.62 & 0.64 & 0.79\\
     & ADC[1] & 1243 & 1064 & 721 & 0.66 & 0.67 & 0.84\\
  M41 & ADC[0] & 1476 & 879 & 973 & 0.77 & 0.78 & 0.90\\
     & ADC[1] & 1238 & 903 & 853 & 0.72 & 0.73 & 0.86\\
  M42 & ADC[0] & 1262 & 934 & 631 & 0.73 & 0.74 & 0.90\\
     & ADC[1] & 1036 & 817 & 648 & 0.68 & 0.72 & 0.87\\
  M43 & ADC[0] & 988 & 1448 & 1336 & 0.46 & 0.49 & 0.60\\
     & ADC[1] & 632 & 676 & 1566 & 0.63 & 0.68 & 0.73\\
  M44 & ADC[0] & 2772 & 521 & 103 & 1.00 & 1.00 & 1.00\\
     & ADC[1] & 2757 & 541 & 123 & 1.00 & 1.00 & 1.00\\
  M45 & ADC[0] & 1193 & 1327 & 962 & 0.56 & 0.55 & 0.69\\
     & ADC[1] & 994 & 1141 & 741 & 0.54 & 0.54 & 0.69\\
  M46 & ADC[0] & 1687 & 1231 & 846 & 0.72 & 0.73 & 0.89\\
     & ADC[1] & 1479 & 899 & 788 & 0.78 & 0.80 & 0.93\\
  M47 & ADC[0] & 1893 & 1036 & 628 & 0.86 & 0.88 & 0.98\\
     & ADC[1] & 1400 & 1000 & 422 & 0.81 & 0.81 & 0.96\\
  M48 & ADC[0] & 2642 & 713 & 345 & 0.99 & 1.00 & 1.00\\
     & ADC[1] & 2780 & 614 & 196 & 1.00 & 1.00 & 1.00\\
  \bottomrule
\end{longtable}

\normalsize
 %\cleardoublepage



    % Bibliography
    \TheBibliography
    %\chapter*{\appendixname} \addcontentsline{toc}{chapter}{\appendixname}
    % BIBTEX
    % use if you want citations to appear even if they are not referenced to:
    % \nocite{*} or maybe \nocite{Kon64,And59} for specific entries
    %\nocite{*}
    \bibliographystyle{alpha}
    \bibliography{lit.bib}

    % THEBIBLIOGRAPHY
    %\begin{thebibliography}{000}
    %    \bibitem{ident}Entry into Bibliography.
    %\end{thebibliography}
\end{document}
