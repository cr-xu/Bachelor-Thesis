The goal of the EDELWEISS-III experiment is to probe the WIMP-nucleon cross-section down to the order of \SI{e-9}{pb}. To achieve such sensitivity, the detectors are enclosed by multiple layers of shielding to reduce backgrounds. The through going muons which can induce WIMP-like signals are tagged with a \mvs{}.

The goal of the this thesis is to investigate the behaviour of the \mvs{} during the time period of seven years since the start of the EDELWEISS-III experiment. The total data obtained by the \mvs{} in seven years from Run70 to Run138 has been analysed.

First, the LED events are analysed for the four extra modules. By comparing change of the mean ADC values from three LEDs on the same module, the contributions of different effects to the aging of module are discussed. The two PMT groups of one module are found to have aged symmetrically. The change of ADC values due to the aging of the scintillator material are derived to be $\approx \SI{3}{channels\per month}$ for Module 8. This value is expected to be higher for other long modules.\\
Secondly, muon events are selected for all 46 modules. A strict cut requiring coincidence in more than two modules with full information is applied to get a clean energy spectrum. The change of MPVs from the measured muon energy deposit are investigated. For the four extra modules, the result is consistent with the one obtained from the LED events. For other long modules, the change of MPV in ADC units is typically $\approx \SI{5}{channels\per month}$.

%threshold blahblah
Due to the low muon rate in underground laboratory, muons cannot be used to determine the trigger threshold of a module. An alternative method is performed using the events that have triggered another module.
However, the position dependent response of modules cannot be derived in this work. A calibration source is needed to determine the position dependency of trigger threshold in individual modules.
Averaging over the module length, the effective trigger threshold is derived for each PMT group. The response is described by an error function parametrized by $E_{\mathrm{thr}}$ and $\sigma$.

The muon energy deposit spectrum is then fitted with a Landau distribution and error function with parameters obtained from the threshold determination. The detection efficiency of each PMT group in different time periods is derived. The obtained values give a lower limit on the detection efficiency of individual modules. For most of the modules, a relative decrease of 30\% is observed.

The sensitivity of WIMP-search can be potentially limited by the $\upmu{}$-induced neutron backgrounds, if the total detection efficiency of the \mvs{} decreases significantly. Therefore, the HVs applied on the modules must be adjusted regularly to ensure that the efficiency of individual module remains stable.
