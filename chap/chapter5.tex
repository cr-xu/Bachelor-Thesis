
% Muon events, threshold analysis
Analysis of LED events can only be applied on the four extra top modules. To determine the stability of the total \mvs, aging of all modules are investigated. In this chapter, the muon-events are selected and analysed. Additionally, the effective threshold of each module is determined using an independent method, which allows to estimate the change of the detection efficiency of the modules.

\section{Analysis of muon events}

\subsection{Selection of muon-veto data}
All the data from Run70 (Aug 2010) to Run138 (Mar 2017) are investigated in the following analysis. Since the goal is to analyse the stability of muon detection in modules, several cuts are applied on the data to ensure a stable condition of the system during the investigated time period and to decrease background events.\\
First, data taken when the \mvs{} was not fully closed are cut. As described in \ref{sec:muon-setup}, two lasers measure the position of the upper part of the muon-veto every 15 minutes. Time periods when the gap size deviates more than 5 cm from the closed configuration are cut, since the \mvs{} is either fully closed or fully opened. When the system is opened, the detection efficiency of through-going muons decreases and the background rate increases largely, which will lead to a increase of low energy events and .

When the \mvs{} is closed, most muons go through at least 2 modules. Therefore a coincidence in 2 distinct modules is required for selecting muon events. An energy deposit with full information in each module is required--i.e.\ both TDCs and ADCs have non-zero values. It respectively reduces events caused by secondary particles or natural backgrounds as they mostly deposit less energy than muons. By applying this cut criterium, the event rate in one module is reduced to the order of $\SI{10}{events/day}$.

\subsection{Data analysis}
As mentioned in section \ref{sec:muon-working}, the spectrum of muon energy deposit in modules can be described by a Landau distribution. The ADC values of muon events in a time period are fitted with a Landau distribution and the obtained MPVs in different time periods are used to analyse the long term behaviour of the \mvs. The fig.\ \ref{fig:Landau_M6} shows an example of the Landau fit. Since the muon event rate is low, the obtained ADC values in a two months period are combined to perform each fit.

\begin{figure}[ht]
  \centering
  \includegraphics[width=0.6\textwidth{}]{./fig/LandauFitM6.pdf}
  \caption{Example of a Landau fit in M6. The fit uses a log likelihood method in ROOT.}
  \label{fig:Landau_M6}
\end{figure}

The MPVs obtained from the Landau fit are plotted over time (see fig.\ \ref{fig:MPV}). Some time periods with too few entries due to the system shutoff are excluded, since no reliable Landau fit can be performed. It is to be noticed that the data points have rather large uncertainty, which is partially due to the smeared spectrum. As explained before, the light output is strongly position dependent and decreases exponentially with the path length from the interaction point to the PMT group. Furthermore, there are some sudden changes of the MPVs. The reason could be the restart of the \mvs{} after operations.

\begin{figure}[hp!]
  \centering
  \begin{subfigure}{0.6\linewidth}
    \centering
    \includegraphics[width=\linewidth{}]{./fig/M6mpv.pdf}
    \caption{}
    \label{fig:MPV_M6}
  \end{subfigure}
  \begin{subfigure}{0.6\linewidth}
    \centering
    \includegraphics[width=\linewidth{}]{./fig/M44mpv.pdf}
    \caption{}
    \label{fig:MPV_M44}
  \end{subfigure}
  \caption{MPVs over time with linear fits in two example modules. The two upper figures show the MPVs in M6 (top module) and the two lower figures show the MPVs in M44 (bottom module).}
  \label{fig:MPV}
\end{figure}

%
%\begin{figure}[htb]
%  \centering
%  \includegraphics[width=0.6\textwidth{}]{./fig/M8mpv.pdf}
%  \caption{MPVs over time with linear fit in M8.}
%  \label{fig:Landau_M8}
%\end{figure}

\begin{table}[hb!]
  \centering
  \caption{Slopes of the linear regressions of MPVs in example modules M6, M8 and M44. The statistical uncertainty is obtained from the fit program in ROOT. }
  \label{tab:mpv}
  \begin{tabular}{c c c c}
  \toprule
  Module & \multicolumn{2}{c}{slope in channels/month} \\
         & ADC N & ADC S \\
  \midrule
  M6  & $-5.44\pm1.34$ & $-4.87\pm1.12$ \\
  M8  & $-5.26\pm0.52$ & $-6.00\pm0.46$ \\
  M44 & $-5.76\pm3.71$ & $-3.71\pm1.12$ \\
  \bottomrule
  \end{tabular}
\end{table}

The change of the MPVs are again approximated by a linear regression and the obtained slopes are listed in tab.\ \ref{tab:mpv}. It can be seen that the slopes for two ends in a single module don't vary much from each other, which means that the aging of plastic scintillator is as expected almost symmetric and the electronics of two PMT groups have also aged similarly.

The value obtained in M8 is also listed for comparison with the result from LED data. The decrease of the MPVs is larger than the one of ADC mean values of middle and northern LED events and lies in the uncertainty regions of southern LED events. The abnormal increase of ADC mean values of LED S in M8 (see tab.\ref{tab:led}) is not to be found in the data obtained from muon events. Thus it can be concluded that the problem is merely due to the behaviour of the LED but not the scintillator module. \\
Despite the various effects that may lead to a sudden change of the ADC value, the MPVs of the Landau spectrum in all 46 modules show a general decrease of several hundred ADC channels since 2010.



\section{Determination of the detection efficiency}
The analysis above shows that the gain of the PMT groups have decreased remarkably since the start of the experiment. However, it doesn't provides a direct measure of the change of the muon detection efficiency. To estimate the detection efficiency, the knowledge of the trigger threshold of each module is needed. Before the experiment, the scintillator modules have been calibrated using cosmic muons \cite{Hab04}. Due to the aging effects of the modules, the trigger threshold must be determined again. As the muon flux in the underground laboratory is reduced significantly, the cosmic muons cannot be used for calibration anymore. In this section, an alternative approach to determine the trigger threshold is described. The detection efficiencies in different time periods are derived using the obtained threshold values.

\subsection{Effective trigger threshold}
As discussed in section \ref{sec:muon-working}, there are two thresholds in the measurement of \mvs. One is a digital conversion threshold, which is typically set to 120\,ADC channels. The influence of digital threshold on the muon detection efficiency can be neglected, because the MPVs are typically above 1000\,ADC channels and the Landau distribution falls quickly towards the low energy region.

The trigger threshold, on the other hand, is relevant for the following analysis. It is a hardware threshold which is set to 150\,mV. Only the pulse with an amplitude above this value triggers the data acquisition. Although the hardware threshold is set to same voltage in each module, the value in ADC channels depends on the gain of the individual PMT groups. Thus the value of trigger threshold is different in each module. \\
Additionally, it depends on the shape of measured pulse, as the ADC value is the integral of a signal. For example, a flat pulse has a larger energy deposit than a steep pulse with same amplitude. The pulse shape also depends on the light propagation and is thus position dependent. However, a calibration source is needed to determine the position dependency of trigger threshold, which is not performed in this work.

Averaging over the module length, the effective trigger threshold is expected to be a curve instead of a step function. The efficiency of a module can be given as a function of energy, which is low for low energy deposit and is expected to be 100\% in high energy region. It is similar to the behaviour of an error function:
\begin{equation}
  \mathrm{erf}(x)=\frac{2}{\sqrt{\pi}} \int_{0}^{x} \! e^{-t^{2}}\, \mathrm{d}t
\end{equation}
It describes the convolution of a Heaviside function $\Theta(x)$ with a Gaussian distribution. By shifting and rescaling, the error function is able to imitate the behaviour of the actual detection efficiency curve at the threshold. It is characterised by two parameters, the effective threshold $E_{\mathrm{thr}}$ and the standard deviation $\sigma$. The effective trigger threshold is defined as the ADC value where detection efficiency is 50\% and the $\sigma$ describes the smearing effect of the spectrum.

\subsubsection*{Method of effective trigger threshold determination}

When a module is triggered, all the non-zero signals in the \mvs{} are stored. It is therefore possible to analyse the the events for a certain module I even if they don't trigger the module I. For the analysis of one PMT group X in module I, the ADC values of all the events that have triggered another module J are stored in histogram H1. In contrast to the condition of muon selection, which requires both non-zero TDC values, the trigger condition of a PMT group X used in the following analysis requires only that the TDC on the same side has a non-zero value. Requiring both TDC channels will introduce a dependency on the PMT group on the other side, while the trigger threshold is determined separately for two ends of a module. The events with one TDC trigger are stored in histogram H2. The efficiency of one PMT group is determined by the division of H2 and H1.

\begin{figure}[ht]
  \centering
  \begin{subfigure}{0.7\linewidth}
    \includegraphics[width=\linewidth{}]{./fig/M6AdcNorth2Histo.pdf}
    \caption{Example of the spectrum with and without trigger condition of ADC N in M6. The data points (blue) are events which satisfy the trigger condition in the same end (Northern PMT group). The spectrum of all events (black) are scaled by a factor of 4.}
    \label{fig:2HistoM6}
  \end{subfigure}
  \begin{subfigure}{0.7\linewidth}
    \includegraphics[width=\linewidth{}]{./fig/M6AdcNortheff_late.pdf}
    \caption{Histogram of events triggering the PMT group on the same side divided by all events with error bars (black) and the error function fit (red).}
    \label{fig:eff_lateM6}
  \end{subfigure}
  \caption{Example of the method to determine the effective trigger threshold of a PMT group.}
  \label{fig:threshold_example}
\end{figure}

Fig.\,\ref{fig:threshold_example} shows an example of this method. The events that triggered one PMT group are plotted with all events. As expected, the detection efficiency increases with ADC values and goes toward 100\%. As can be seen in the figure, the increase is slightly asymmetric: it is more rapid at first and is then flatter. The estimation with an error function is therefore conservative for low energy events and optimistic for high energy events.

The effective trigger threshold in ADC units are calculated for each PMT group in different time periods. To increase statistics, several runs are combined to perform the error function fit. Run 70-79 (year 2010) are investigated for the early time period and Run 124-138 (year 2015-2017) for the late time period. The results are listed in tab. \ref{tab:efficiency_short} in the following section.


\subsection{Detection efficiency of a module}
In addition to the above criteria, events with a coincidence in adjacent modules are excluded. These events have a higher probability to be induced by secondary particles or grazing muons. They mostly have low energy deposits, where the efficiency is also low due to the trigger threshold, and can lead to a decrease of total detection efficiency.

\begin{figure}[h!]
  \begin{subfigure}{0.5\linewidth}
    \includegraphics[width=\linewidth{}]{./fig/70M6CorrectedLandau.pdf}
    \caption{}
    \label{fig:landau_earlyM6}
  \end{subfigure}
  \begin{subfigure}{0.5\linewidth}
    \includegraphics[width=\linewidth{}]{./fig/124M6CorrectedLandau.pdf}
    \caption{}
    \label{fig:landau_lateM6}
  \end{subfigure}
  \caption{Spectrum of muon-induced events in M6 in earlier runs (left) and latest runs (right). The spectra are fitted with a Landau distribution with an error function (red). The corrected spectrum (black, dotted) is determined by dividing the error function.}
  \label{fig:detection_M6}
\end{figure}

The muon energy deposit can be described with a Landau distribution, whereas the measured energy deposit is dependent on the response of the individual modules. Therefore, the ADC spectrum is expected to be a Landau multiplied by an error function. The selected data are fitted with this function and the corrected spectrum is given by a Landau distribution with the obtained parameters from the fit (see fig. \ref{fig:detection_M6}). The efficiency can be determined by the integral of the spectrum with and without the error function. The results for selected modules are listed in tab \ref{tab:efficiency_short}, the complete table is given in appendix \ref{sec:appendix}.


\begin{table}[htb!]
  \caption{The trigger threshold and detection efficiency of PMT groups in selected modules on different sides of the \mvs{} (for the full table see appendix \ref{sec:appendix}). The MPV in this table denotes the most probable value of the Landau distribution with error function. The total efficiency is estimated by the product of the efficiency value $\epsilon_{50\%\mathrm{MPV}}$ of each PMT group, which is determined by the integration from half of the MPV energy. }
  \label{tab:efficiency_short}
  \begin{tabular}{c c c c c c c c c}
    \toprule
    Module & End & MPV & Threshold & $\sigma{}_{\mathrm{erf}}$ & $\epsilon_{20\%}$ & $\epsilon_{50\%\mathrm{MPV}}$ & $\epsilon_{\mathrm{MPV}}$ & $\epsilon_{\mathrm{tot}, 50\%\mathrm{MPV}}$ \\
           &     & \multicolumn{3}{|c|}{in ADC channels} &   \\
    \midrule
    M6     & N & 1401 & 532 & 737 & 0.86 & 0.90 & 0.97 & \multirow{2}{*}{0.77}\\
    early  & S & 1256 & 615 & 712 & 0.83 & 0.86 & 0.95 &\\
    M6     & N & 872 & 707 & 627 & 0.67 & 0.71 & 0.85 & \multirow{2}{*}{0.53}\\
    late   & S & 975 & 746 & 416 & 0.60 & 0.75 & 0.94\\
    \midrule
    M15    & Nemo & 2674 & 560 & 143 & 0.99 & 1.00 & 1.00 & \multirow{2}{*}{1.00}\\
    early  & Est & 1292 & 186 & 13 & 1.00 & 1.00 & 1.00\\
    M15    & Nemo & 1713 & 1879 & 758 &  &  &  & \multirow{2}{*}{0.59}\\
    late   & Est & 1284 & 958 & 280 & 0.73 & 0.77 & 0.99 & \\
    \midrule
    M17    & B & 1200 & 515 & 112 & 0.99 & 1.00 & 1.00 & \multirow{2}{*}{0.94}\\
    early  & H & 1043 & 100 & 740 & 0.88 & 0.94 & 0.98 &\\
    M17    & B & 1457 & 855 & 384 & 0.82 & 0.90 & 0.99 & \multirow{2}{*}{0.80} \\
    late   & H & 1359 & 671 & 611 & 0.72 & 0.89 & 0.98 & \\
    \midrule
    M36    & Est & 1452 & 912 & 803 & 0.61 & 0.80 & 0.93 & \multirow{2}{*}{0.67} \\
    early  & Nemo & 1346 & 821 & 568 & 0.79 & 0.83 & 0.96 &\\
    M36    & Est & 1201 & 1114 & 624 & 0.48 & 0.62 & 0.85 & \multirow{2}{*}{0.45}\\
    late   & Nemo & 1373 & 1060 & 531 & 0.61 & 0.72 & 0.93 & \\
    \bottomrule
  \end{tabular}
\end{table}

For each PMT group, three efficiency values are given. $\epsilon_{20\%}$ integrates from the energy where the entries are equal to 20\% of the entries at MPV. $\epsilon_{\mathrm{MPV}}$ integrates from the MPV and gives a optimistic estimation. Therefore, the total efficiency of a module is given by the product of $\epsilon_{50\%\mathrm{MPV}}$ in two PMT groups, for which the spectrum is integrated from half of the energy at MPV.

For some of the PMT groups, the spectrum cannot be fitted properly (for example fig.\ref{fig:fail_M15}). The efficiency is estimated by the other PMT group in the same module. If the fit method cannot be applied for both PMT groups, the total detection efficiency is given by the mean values of other modules on the same side.


\begin{figure}[ht!]
  \centering
  \includegraphics[width=0.7\textwidth{}]{./fig/M15fail.pdf}
  \caption{Example of a failed fit of PMT Nemo in M15 in Run124-138. The efficiency is given by the other PMT group in the same module.}
  \label{fig:fail_M15}
\end{figure}

As expected, the MPV decreased and the trigger threshold increased in the time period of seven years, which lead to a loss of detection efficiency. Most modules show a relative decrease of 30\%. Although the relative change is significant, the absolute values of detection efficiency should only be used as lower limits, since they are strongly underestimated. An analysis in 2010 using similar method obtained the result $\epsilon_{\mathrm{M6}}=0.88$ and $\epsilon_{\mathrm{M36}}=0.96$ \cite{Nie10}, showing that the efficiency values derived here are indeed too conservative.

The underestimation results from the following facts. First of all, there are still contributions of secondary particles or backgrounds to the low energy region despite the cuts applied. The obtained MPVs from fit are thus smaller and it leads to a smaller detection efficiency.

Secondly, the gain of two PMT groups are considered as uncorrelated for the matter of simplification, while they are correlated in reality. As shown in section \ref{sec:muon-working}, the light yield measured in near PMT group is larger than the one in far PMT group. This leads to an overestimation for near end and underestimation for far end. Since the efficiency decreases drastically in low energy region, the total detection efficiency is expected to be underestimated. However, the effect of correlation cannot be implemented due to the low statistics of muon events.

%why is eff so low

Additionally, the muons passing through the \mvs{} have a second chance to be detected when leaving the system. Also, they can partly be detected by measuring the particle showers. Last but not least, the grazing muons, which go through adjacent modules and deposit energy well below MPV, are possible to be detected in both modules. To conclude, the absolute value of detection efficiency derived here should be treated as lower limit and the total efficiency of the \mvs{} is higher as the one of individual modules.

The result shows that the detection efficiency for most of the modules has decreased significantly. Despite that the rate of $\upmu{}$-induced WIMP-like signals is estimated to be low, it can limit the sensitivity of WIMP search if the efficiency of \mvs{} drops. To retain a high detection efficiency as the start of the experiment, the HV applied on individual PMT groups should be increased. By doing so, the gain of PMT groups increases, corresponding to a higher MPV in ADC units and lower effective threshold.




%blahblah
