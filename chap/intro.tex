Astrophysical and cosmological observations over the last decades indicate the existence of some non-baryonic dark matter. By analyzing the anisotropy of cosmic microwave background, the dark matter energy contribution is estimated to be 27\% of the universe \cite{Pla16}. Yet no knowledge of the particle constituent of the dark matter is obtained.

A generic class of hypothetical particles, the Weakly Interacting Massive Particle (WIMP), is a prominent candidate for the dark matter. WIMPs are often assumed to have a mass of $\mathcal{O}$(\SI{100}{GeV}), with an interaction cross section with ordinary matter of the order of the weak interaction scale.

The EDELWEISS experiment aims to search for a direct signal of elastic scattering of WIMPs on germanium nuclei. Due to the expected low rate of WIMP-nucleus scattering, the main challenge of the experiment is to understand and exclude potentially all the background events.

The detectors are surrounded by multiple layers of external shielding, which absorb and reject background radioactivity. Further backgrounds from the radioactivity of the shielding materials can be discriminated by the simultaneous readout of heat and ionization signals. The remaining neutron background causes a nuclear recoil in detectors, which can hardly be distinguished from a WIMP-signal. The neutrons are produced either by $(\upalpha,\mathrm{n})$ reaction from natural radioactivity or by cosmic-ray muons and natural radioactivity. To protect the detectors from the cosmic muon background, EDELWEISS is located in the underground laboratory in Modane (Laboratoire Souterrain de Modane, LSM), where the muon flux is reduced to 5\,$\mathrm{\mu}$/$\mathrm{m}^{2}$/d \cite{Sch13a}. The remaining muons are tagged by a \mvs{} of 46 plastic scintillator modules.

Since the start of the EDELWEISS experiment, the scintillator modules as well as the electronics have aged significantly. The goal of this work is to estimate the stability of the muon veto system over a period of several years. Four extra scintillator modules installed in 2010 are also equipped with LEDs allowing to monitor the stability of the system with this special light source.

In chapter \ref{chap:theo} the case of dark matter with focus on WIMPs is discussed, followed by a brief description of the general setup of the EDELWEISS experiment. In chapter \ref{chap:muon} the setup and the working principle of the \mvs is described. First, in chapter \ref{chap:ana_led} the LED events are analysed to estimate the long term stability of these four modules. Second, in chapter \ref{chap:ana_muon} the muon events are selected for the analysis of all modules. Additionally, the effective threshold of each module is determined to estimate the change of the detection efficiency of the \mvs.
