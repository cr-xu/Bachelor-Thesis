Astrophysical and cosmological observations over the last decades indicate the existence of some none-baryonic dark matter. By analyzing the anisotropy of cosmic microwave bacground, the dark matter energy contribution is estimated to be 27\% of the universe. Yet no knowledge of the particle consitituent of the dark matter is obtained.

A generic class of hypothetical particles, the Weakly Interacting Massive Particle (WIMP), is a prominent candidate for the dark matter. WIMPs is often assumed to have mass of $\mathcal{O}$(\SI{100}{GeV}), with an interaction cross section with ordinary matter of the order of weak interaction scale.

The EDELWEISS experiment is aimed to search direct signal of elastic scattering of WIMP on germanium bolometers. Due to the expected low rate of WIMP-nucleus scattering, the main challenge of the experiment is to understand and exclude possibly all the background events.

The detectors are surrounded by multiple layers of external shielding, which absorb and reject bacground radioactivity. Further backgrounds from the radioactivity of the shieding materials can be discriminated by the simultaneous readout of heat and ionization measurements. (\underline(cite))  The remaining neutron bacground causes a nuclear recoil in detectors, which cannot be distinguished from a WIMP-signal. The neutrons are produced respectively from the cosmic-ray muons. To protect the detectors from the cosmic muon bacgrounds, EDELWEISS is located in the underground laboratory in Modane (Laboratoire Souterrain de Modane,LSM), where the muon flux is reduced to 5\,$\mathrm{\mu}$/$\mathrm{m}^{2}$/d,(!cite). The remaining muons are tagged by a muon-veto system of 46 plastic scintillator modules.

Since the start of EDELWEISS experiment, the modules as well as the electronics have decayed significantly. The goal of this presented work is to estimate the stability of the muon veto system over long term measurements. Four scintillator modules are equipped with LEDs to moniter the stability of the system. The LED induced events are first analysed to determine the stability of these modules. Muon events are selected and analysed for all 46 modules.

In chapter \ref{chap:theo} the case of dark matter with focus on WIMPs is discussed.
