Astrophysical and cosmological observations over the last decades indicate the existence of some non-baryonic dark matter. By analyzing the anisotropy of cosmic microwave bacground, the dark matter energy contribution is estimated to be 27\% of the universe \cite{Ade16}. Yet no knowledge of the particle consitituent of the dark matter is obtained.

A generic class of hypothetical particles, the Weakly Interacting Massive Particle (WIMP), is a prominent candidate for the dark matter. WIMPs are often assumed to have a mass of $\mathcal{O}$(\SI{100}{GeV}), with an interaction cross section with ordinary matter of the order of the weak interaction scale.

The EDELWEISS experiment aims to search for a direct signal of elastic scattering of WIMP on germanium nuclei. Due to the expected low rate of WIMP-nucleus scattering, the main challenge of the experiment is to understand and exclude possibly all the background events.

The detectors are surrounded by multiple layers of external shielding, which absorb and reject bacground radioactivity. Further backgrounds from the radioactivity of the shieding materials can be discriminated by the simultaneous readout of heat and ionization measurements. The remaining neutron bacground causes a nuclear recoil in detectors, which cannot be distinguished from a WIMP-signal. The neutrons are produced respectively from the cosmic-ray muons and natural radioactivity. To protect the detectors from the cosmic muon bacgrounds, EDELWEISS is located in the underground laboratory in Modane (Laboratoire Souterrain de Modane,LSM), where the muon flux is reduced to 5\,$\mathrm{\mu}$/$\mathrm{m}^{2}$/d \cite{Sch13a}. The remaining muons are tagged by a \mvs of 46 plastic scintillator modules.

Since the start of EDELWEISS experiment, the modules as well as the electronics have decayed significantly. The goal of this presented work is to estimate the stability of the muon veto system over long term measurements. Four extra scintillator modules installed in 2010 are equipped with LEDs to moniter the stability of the system.

In chapter \ref{chap:theo} the case of dark matter with focus on WIMPs is discussed, followed by a brief description of the general setup of EDELWEISS experiment. In chapter \ref{chap:muon} the setup and the working principle of \mvs is described. First, the LED events are analysed to estimate the long term stability of these four modules in chapter \ref{chap:ana_led}. Second, the muon events are selected for the analysis of all modules in chapter \ref{chap:ana_muon}. Additionaly, the effective threshold of each module is determined to estimate the change of detection efficiency of the \mvs.
