\documentclass{include/thesisclass}
% Main File - Based on thesisclass.cls
% Comments are mostly in English
% ------------------------------------------------------------------------------
% Further files in folder:
%  - include/cmds.tex (for macros and additional commands)
%  - include/kitlogo.pdf (for titlepage)
%  - lit.bib (bibtex bibliography database)
%  - include/titlepage.tex (for layout of titelpage)
% ------------------------------------------------------------------------------
% Useful Supplied Packages:
% amsmath, amssymb, mathtools, bbm, upgreek, nicefrac,
% siunitx, varioref, booktabs, graphicx, tikz, multicol





%% -------------------------
%% |    Thesis Settings    |
%% -------------------------
% english or ngerman (new german für neue deutsche Rechtschreibung statt german)
\SelectLanguage{english}
% details on this thesis
\newcommand{\thesisauthor}{Chenran Xu}
\newcommand{\thesistopic}{Untersuchung der Langzeitstabilität des EDELWEISS Myon-Veto-Systems}
\newcommand{\thesisentopic}{Investigation of the long term stability of the EDELWEISS muon veto system}
\newcommand{\thesislongtopic}{Very long and very detailed description of the very interesting thesis topic (only necessary for pdfsubject tag).}
\newcommand{\thesisinstitute}{Institut für Experimentelle Kernphysik}
\newcommand{\thesisreviewerone}{Prof. Dr. D. Cay}
\newcommand{\thesisreviewertwo}{Prof. Dr. E. Vil}
\newcommand{\thesisadvisorone}{} % to use: enter names and uncomment in titlepg
\newcommand{\thesisadvisortwo}{}
\newcommand{\thesistimestart}{01.04.2015} % on titlepage
\newcommand{\thesistimeend}{30.09.2015} % on titlepage
\newcommand{\thesistimehandin}{30.09.2015} % on second page 'preamble'
\newcommand{\thesispagehead}{Bachelor Thesis: \thesisentopic} % page heading





%% ---------------------
%% |    PDF - Setup    |
%% ---------------------
% This information will appear embed into the PDF file as meta data, but will
% not be printed anywhere
\hypersetup
{
    pdfauthor={\thesisauthor},
    pdftitle={Bachelorarbeit: \thesistopic},
    pdfsubject={\thesislongtopic},
    pdfkeywords={kit,physik,bachelor,thesis,\thesisauthor}
}





%% --------------------------------------
%% |    Settings for Word Separation    |
%% --------------------------------------
% Help for separation:
% In German package the following hints are additionally available:
% "- = Additional separation
% "| = Suppress ligation and possible separation (e.g. Schaf"|fell)
% "~ = Hyphenation without separation (e.g. bergauf und "~ab)
% "= = Hyphenation with separation before and after
% "" = Separation without a hyphenation (e.g. und/""oder)

% Describe separation hints here:
\hyphenation
{
    über-nom-me-nen an-ge-ge-be-nen
    %Pro-to-koll-in-stan-zen
    %Ma-na-ge-ment  Netz-werk-ele-men-ten
    %Netz-werk Netz-werk-re-ser-vie-rung
    %Netz-werk-adap-ter Fein-ju-stier-ung
    %Da-ten-strom-spe-zi-fi-ka-tion Pa-ket-rumpf
    %Kon-troll-in-stanz
}





%% -----------------------
%% |    Main Document    |
%% -----------------------
\usepackage{lipsum} % for Lorem Ipsum text example
\begin{document}
    % Titlepage and ToC
    \FrontMatter

    \input{include/titlepage}
    \input{include/preamble}

    \begingroup \let\clearpage\relax    % in order to avoid listoffigures and
    \tableofcontents                    % listoftables on new pages
    \listoffigures
    \listoftables
    \endgroup
    \cleardoublepage



    % Contents
    \MainMatter

    \chapter{Introduction}
    some introduction &
    outline
    \chapter{WIMPs-search with EDELWEISS-III Experiment}
      \section{Evidences of dark matter}
      kinematic, interaction.
      \section{WIMP as dark matter candidate}

      \section{EDELWEISS-III Experiment}
        \subsection{Experimental setup}
        \subsection{Backgrounds at EDELWEISS Experiment}

    \chapter{Detection of muons in EDELWEISS-III Experiment}
      \section{Muon veto system}
        \subsection{Experimental setup}
        \subsection{Energy deposit of muon in the scintillator module}
        \subsection{Working Principle}
          electronics

    \chapter{Analysis of the long term measurement data}
      \section{(Determination of the aging effect using) LED events}
        \subsection{Data selection}
          cut criterium
        \subsection{}

      \section{Muon events}
        \subsection{Data selection}
        \subsection{Analysis of MPV of Landau-Specturm}
      \section{Muon Detection Efficiency}
        \subsection{Effective trigger threshold}
         methods used,
    \chapter{Conclusions}




    %\emptychapter[3]{ROOT Routines}     % usage: \emptychapter[page displayed
                                        %        in toc]{name of the chapter}


    % appendix for more or less interesting calculations
    \Appendix
    \chapter*{\appendixname} \addcontentsline{toc}{chapter}{\appendixname}
    % to make the appendix appear in ToC without number. \appendixname =
    % Appendix or Anhang (depending on chosen language)
    % \section{Detection Efficiency}
some text!
%tables of efficiency
\small
\begin{longtable}{c c c c c c c c c}
  \caption{Slopes of the linear regressions of MPVs in example modules M6 and M44. The statistical uncertainty is obtained from the fit program in ROOT. } \\
  %\label{tab:mpv-full}
  \toprule
  Module & End & MPV & Threshold & $\sigma{}_{\mathrm{erf}}$ & $\epsilon_{20\%}$ & $\epsilon_{50\%\mathrm{MPV}}$ & $\epsilon_{\mathrm{MPV}}$ & $\epsilon_{\mathrm{tot}, 50\%\mathrm{MPV}}$ \\
         &     & \multicolumn{3}{|c|}{in ADC channels} &   \\
  \midrule
  \endfirsthead
  M1 & ADC[0] & 1036 & 939 & 690 & 0.66 & 0.65 & 0.80 & \multirow{2}{*}{0.4}\\
     & ADC[1] & 827 & 191 & 2193 & 0.69 & 0.71 & 0.75 & \\
  M2 & ADC[0] & 625 & 786 & 1622 & 0.56 & 0.60 & 0.66\\
     & ADC[1] & 214 & 829 & 1905 & 0.53 & 0.55 & 0.57\\
  M3 & ADC[0] & 647 & 494 & 98 & 0.88 & 0.83 & 1.00\\
     & ADC[1] & 918 & 635 & 136 & 0.90 & 0.90 & 1.00\\
  M4 & ADC[0] & 352 & 880 & 194 & 0.01 & 0.01 & 0.01\\
     & ADC[1] & 355 & 890 & 157 & 0.00 & 0.00 & 0.00\\
  M5 & ADC[0] & 1197 & 554 & 865 & 0.79 & 0.84 & 0.93\\
     & ADC[1] & 1108 & 645 & 607 & 0.78 & 0.83 & 0.94\\
  \midrule
  M6 & ADC[0] & 1401 & 532 & 737 & 0.86 & 0.90 & 0.97\\
     & ADC[1] & 1256 & 615 & 712 & 0.83 & 0.86 & 0.95\\
  M7 & ADC[0] & 1080 & 190 & 366 & 0.97 & 0.99 & 1.00\\
     & ADC[1] & 941 & 202 & 223 & 0.98 & 0.99 & 1.00\\
  M8 & ADC[0] & 1443 & 638 & 155 & 0.99 & 0.99 & 1.00\\
     & ADC[1] & 2217 & 251 & 101 & 0.98 & 1.00 & 1.00\\
  M9 & ADC[0] & 1320 & 751 & 791 & 0.71 & 0.82 & 0.93\\
     & ADC[1] & 3000 & 100 & 1282 & 0.86 & 0.97 & 1.00\\
  M10 & ADC[0] & 650 & 476 & 910 & 0.68 & 0.76 & 0.84\\
     & ADC[1] & 1410 & 100 & 989 & 0.87 & 0.93 & 0.98\\
  \midrule
  M11 & ADC[0] & 1133 & 685 & 409 & 0.82 & 0.86 & 0.98\\
     & ADC[1] & 952 & 493 & 176 & 0.94 & 0.97 & 1.00\\
  M12 & ADC[0] & 881 & 660 & 120 & 0.92 & 0.86 & 1.00\\
     & ADC[1] & 1140 & 395 & 146 & 0.97 & 1.00 & 1.00\\
  M13 & ADC[0] & 799 & 122 & 1184 & 0.79 & 0.85 & 0.90\\
     & ADC[1] & 1331 & 298 & 784 & 0.84 & 0.94 & 0.98\\
  M14 & ADC[0] & 1118 & 836 & 489 & 0.73 & 0.75 & 0.92\\
     & ADC[1] & 1138 & 100 & 1729 & 0.77 & 0.82 & 0.87\\
  M15 & ADC[0] & 2674 & 560 & 143 & 0.99 & 1.00 & 1.00\\
     & ADC[1] & 1292 & 186 & 13 & 1.00 & 1.00 & 1.00\\
  \midrule
  M16 & ADC[0] & 2481 & 267 & 91 & 0.99 & 1.00 & 1.00\\
     & ADC[1] & 2451 & 296 & 103 & 0.98 & 1.00 & 1.00\\
  M17 & ADC[0] & 1200 & 515 & 112 & 0.99 & 1.00 & 1.00\\
     & ADC[1] & 1043 & 100 & 740 & 0.88 & 0.94 & 0.98\\
  M18 & ADC[0] & 893 & 692 & 1254 & 0.65 & 0.69 & 0.78\\
     & ADC[1] & 2184 & 517 & 153 & 0.97 & 1.00 & 1.00\\
  M19 & ADC[0] & 1207 & 587 & 146 & 0.97 & 0.98 & 1.00\\
     & ADC[1] & 1799 & 533 & 146 & 0.99 & 1.00 & 1.00\\
  M20 & ADC[0] & 2133 & 1352 & 1343 & 0.55 & 0.72 & 0.87\\
     & ADC[1] & 1476 & 687 & 639 & 0.74 & 0.90 & 0.98\\
  M21 & ADC[0] & 1117 & 701 & 841 & 0.65 & 0.79 & 0.91\\
     & ADC[1] & 1219 & 529 & 1024 & 0.73 & 0.84 & 0.92\\
  M22 & ADC[0] & 442 & 1156 & 349 & 0.09 & 0.08 & 0.14\\
     & ADC[1] & 396 & 1008 & 227 & 0.01 & 0.01 & 0.01\\
  M25 & ADC[0] & 349 & 100 & 745 & 0.78 & 0.83 & 0.87\\
     & ADC[1] & 1261 & 708 & 1196 & 0.69 & 0.77 & 0.87\\
  M26 & ADC[0] & 488 & 355 & 866 & 0.70 & 0.77 & 0.83\\
     & ADC[1] & 1202 & 1005 & 1117 & 0.56 & 0.70 & 0.82\\
  M27 & ADC[0] & 1139 & 1110 & 810 & 0.60 & 0.61 & 0.77\\
     & ADC[1] & 1035 & 1104 & 941 & 0.58 & 0.58 & 0.71\\
  M28 & ADC[0] & 339 & 879 & 1238 & 0.51 & 0.55 & 0.59\\
     & ADC[1] & 958 & 1075 & 856 & 0.45 & 0.60 & 0.76\\
  M29 & ADC[0] & 696 & 1289 & 836 & 0.31 & 0.41 & 0.56\\
     & ADC[1] & 1334 & 1253 & 1137 & 0.59 & 0.62 & 0.76\\
  M30 & ADC[0] & 1166 & 1269 & 998 & 0.56 & 0.57 & 0.70\\
     & ADC[1] & 816 & 1054 & 1184 & 0.52 & 0.57 & 0.67\\
  M31 & ADC[0] & 1142 & 1164 & 847 & 0.58 & 0.59 & 0.75\\
     & ADC[1] & 504 & 969 & 1574 & 0.54 & 0.58 & 0.63\\
  M32 & ADC[0] & 466 & 1387 & 1456 & 0.41 & 0.46 & 0.51\\
     & ADC[1] & 340 & 1338 & 1263 & 0.37 & 0.41 & 0.46\\
  M33 & ADC[0] & 1031 & 593 & 264 & 0.86 & 0.93 & 1.00\\
     & ADC[1] & 1404 & 755 & 321 & 0.91 & 0.94 & 1.00\\
  M34 & ADC[0] & 1091 & 1459 & 1129 & 0.39 & 0.52 & 0.67\\
     & ADC[1] & 1157 & 1404 & 852 & 0.49 & 0.49 & 0.66\\
  M35 & ADC[0] & 1125 & 1723 & 1192 & 0.41 & 0.42 & 0.54\\
     & ADC[1] & 1016 & 1557 & 1039 & 0.42 & 0.41 & 0.52\\
  M36 & ADC[0] & 1452 & 912 & 803 & 0.61 & 0.80 & 0.93\\
     & ADC[1] & 1346 & 821 & 568 & 0.79 & 0.83 & 0.96\\
  M37 & ADC[0] & 1411 & 730 & 505 & 0.77 & 0.90 & 0.99\\
     & ADC[1] & 1329 & 746 & 425 & 0.85 & 0.89 & 0.99\\
  M38 & ADC[0] & 1446 & 1247 & 1167 & 0.61 & 0.65 & 0.79\\
     & ADC[1] & 1223 & 1293 & 700 & 0.54 & 0.54 & 0.75\\
  M39 & ADC[0] & 1570 & 967 & 734 & 0.74 & 0.81 & 0.95\\
     & ADC[1] & 1156 & 876 & 484 & 0.75 & 0.75 & 0.92\\
  M40 & ADC[0] & 1208 & 1098 & 913 & 0.62 & 0.64 & 0.79\\
     & ADC[1] & 1243 & 1064 & 721 & 0.66 & 0.67 & 0.84\\
  M41 & ADC[0] & 1476 & 879 & 973 & 0.77 & 0.78 & 0.90\\
     & ADC[1] & 1238 & 903 & 853 & 0.72 & 0.73 & 0.86\\
  M42 & ADC[0] & 1262 & 934 & 631 & 0.73 & 0.74 & 0.90\\
     & ADC[1] & 1036 & 817 & 648 & 0.68 & 0.72 & 0.87\\
  M43 & ADC[0] & 988 & 1448 & 1336 & 0.46 & 0.49 & 0.60\\
     & ADC[1] & 632 & 676 & 1566 & 0.63 & 0.68 & 0.73\\
  M44 & ADC[0] & 2772 & 521 & 103 & 1.00 & 1.00 & 1.00\\
     & ADC[1] & 2757 & 541 & 123 & 1.00 & 1.00 & 1.00\\
  M45 & ADC[0] & 1193 & 1327 & 962 & 0.56 & 0.55 & 0.69\\
     & ADC[1] & 994 & 1141 & 741 & 0.54 & 0.54 & 0.69\\
  M46 & ADC[0] & 1687 & 1231 & 846 & 0.72 & 0.73 & 0.89\\
     & ADC[1] & 1479 & 899 & 788 & 0.78 & 0.80 & 0.93\\
  M47 & ADC[0] & 1893 & 1036 & 628 & 0.86 & 0.88 & 0.98\\
     & ADC[1] & 1400 & 1000 & 422 & 0.81 & 0.81 & 0.96\\
  M48 & ADC[0] & 2642 & 713 & 345 & 0.99 & 1.00 & 1.00\\
     & ADC[1] & 2780 & 614 & 196 & 1.00 & 1.00 & 1.00\\
  \bottomrule
\end{longtable}

\normalsize
 %\cleardoublepage



    % Bibliography
    \TheBibliography

    % BIBTEX
    % use if you want citations to appear even if they are not referenced to:
    % \nocite{*} or maybe \nocite{Kon64,And59} for specific entries
    %\nocite{*}
    \bibliographystyle{babalpha}
    \bibliography{lit.bib}

    % THEBIBLIOGRAPHY
    %\begin{thebibliography}{000}
    %    \bibitem{ident}Entry into Bibliography.
    %\end{thebibliography}
\end{document}
